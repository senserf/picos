\documentstyle [amssymbols,11pt] {article}
\input {../../../MANUAL/rm_en}

\begin{document}

\begin{titlepage}

\vspace*{3.5in}
\begin{center}
\Huge\bf HUBNET \vspace{0.25in}
\end{center}
\end{titlepage}

\section{The protocol}

{\em Hubnet\/} consists of a number
of regular stations connected to a central
switching device---the so-called {\em Hub}.
Each regular station is connected to the {\em Hub} via
two channels: the broadcast channel and the selection channel.
All the broadcast channels are connected
together and form a uniform broadcast medium (a single link).
The selection channels are
not connected: each selection channel is a separate link attaching one
regular station to the {\em Hub}.

The broadcast channel can be in one of two states which are recognized and
toggled by the {\em Hub}.
These states are: {\em busy\/}, meaning that a packet is being transmitted
on the broadcast medium, and {\em idle\/}, when, according to the {\em Hub}'s
perception, the broadcast channel is inactive.

A station willing to
transmit a packet sends it on the selection channel.
When the packet arrives at the {\em Hub},
the {\em Hub} determines the state of the broadcast link.
If the broadcast link
is {\em idle}, the {\em Hub} connects the selection channel
to the broadcast link and marks the broadcast link as {\em busy}.
Thus, the packet is relayed on the
broadcast channel and it will be heard by all stations in due time.
As soon as the transmission is complete, i.e., 
the {\em Hub} detects silence on the selection channel,
the channel is disconnected from the broadcast link and
the broadcast link status is changed to {\em idle}.

If a packet arrives on a selection link while the broadcast channel is
{\em busy}, the packet is simply ignored.
The transmitting
station listens for the echo of its packet on the broadcast channel.
If the echo does not arrive after some time (depending on the distance
of the station from the {\em Hub}), the station assumes that its transmission
has been blocked and tries again.

\section{Startup and termination}

The {\tt Root} process is responsible for building the network model,
creating the protocol processes, and detecting the end of simulation.
The process has two states.
The first state ({\tt Start}) is triggered automatically when the process is
created.
At this state, the process sets up the relation between the {\em ETU\/} and
{\em ITU} (in this case one {\em ETU\/} is declared to be equal to $1000$
{\em ITU\/}s), defines the simulation exit condition (the simulation will stop
as soon as $200$ messages are completely received at their destinations or
the simulated time reaches $1,000,000,000$ {\em ITU}s, whichever happens first),
and calls three user-defined functions.

The first of these functions ({\tt initTopology})
builds the network, the second ({\tt initTraffic}) defines
the traffic conditions, and the third ({\tt initProtocol})
creates the protocol processes.
Then, {\tt Root} awaits for the {\tt DEATH} of the internal process
{\tt Kernel}.
This event marks the end of the simulation run and when it happens
the {\tt Root} process is restarted at state {\tt Exit}.

By invoking the {\tt printPfm} method of the only traffic
pattern created by {\tt initTraffic}, the {\tt Root} process prints out
the standard set of performance measures associated with the traffic pattern.
Then, the process terminates itself and the simulation run is actually
finished.

\section{Stations and processes}

The network consists of two different types of stations: homogeneous
regular stations represented by type {\tt User},
and the {\em Hub} (type {\tt Hub}) acting as a switching device.
Each regular station has two ports: one to the broadcast link ({\tt BPort}),
the other to the selection link ({\tt SPort}).
Moreover, three mailboxes are defined to be used for synchronization
of the protocol processes run at the station.
The ports and mailboxes are created by the station's {\tt setup} method,
immediately after the station itself has been created.

One more attribute of a regular station is a packet buffer ({\tt Buffer})
used to store packets acquired from the client for transmission.
As this attribute is declared statically (as a structure, not a pointer)
within the
station class, it doesn't have to be created explicitly.
Note that the ports and mailboxes could also be declared statically.
Similarly, the packet buffer could be declared as a pointer and then created
dynamically in the {\tt User} setup method.

The {\tt Hub} has one port to the broadcast link ({\tt BPort}) and
{\tt NUsers} ports to the selection links (array {\tt SPorts}),
where {\tt NUsers} is the number of regular stations.

Similarly as for a regular station, the {\tt Hub} ports
are created upon the {\tt setup} of the {\tt Hub}.
Note that the transmission rate of a {\tt Hub}'s selection port is
irrelevant (no {\tt create} argument is given when the selection ports are
created);
so is the transmission rate of a regular station's broadcast port.
One additional attribute off type
{\tt Hub} is the status flag ({\tt Status}) which can
assume one of two values: {\tt Idle} and {\tt Busy}.

Each regular station runs three ``permanent" processes and occasionally
spawns one additional process which disappears after a while.

The {\tt Xmitter} process transmits packets on the station's selection port
and determines whether the transmission has been successful.
The two local attributes of the process, {\tt SPort} and {\tt Buffer}, are set
by the {\tt setup} function to reference directly
the station's selection port and the only packet buffer, respectively.
Alternatively, these items could be referenced as {\tt S->SPort} and
{\tt S->Buffer}.

The {\tt Receiver} process takes care of the station's broadcast port.
It awaits on that port packets addressed to the station owning the process.

The {\tt Monitor} process communicates with {\tt Xmitter} and notifies it
about the success or failure of its last transfer attempt.
The process monitors the broadcast port at its station and detects packets
transmitted by the station.

{\tt Trumpet} is a ``sidekick" process for the {\tt Monitor}.
It models an independent alarm clock which goes off after {\tt EchoTimeout}
time units---to notify the {\tt Monitor} that the echo waiting time has
elapsed.

The {\tt Hub} station runs {\tt NUsers} identical copies of the same process,
each copy servicing one selection port.
Each of the {\tt Hub} processes has access to one selection port
and the broadcast port.
The index of the selection port serviced by the process is specified
as the {\tt setup} parameter when the process is created.
Note that this index is equal to the {\tt Id} of the station connected
to the other side of the selection link plus one (see below).

\section{Topology definition}

The network topology definition
starts with reading in the number of regular stations ({\tt NUsers})
the length of link segment connecting a regular station with the {\tt Hub}
(we assume that all these segments are of the same length), and the
transmission rate (common for all ports).

Then the broadcast link (with {\tt NUsers}$+1$ ports)
and the {\tt Hub} station are created.
Note that the {\tt Hub} is the first created station and its {\tt Id} is
$0$.
Thus, the regular stations are numbered from $1$ to {\tt NUsers}.
The broadcast port of the {\tt Hub} is connected to the broadcast link.

The first {\tt for} loop creates the {\tt NUser} selection links and
regular stations.
Each selection link has two ports.
In this loop we also connect the two station ports to the appropriate links
and connect the other end of the selection link to the {\tt Hub}.

The next two loops assign distances between all pairs of ports of each
link.
The outer loop executes for all regular stations: it defines the distance
between a regular station and the {\tt Hub}.
The inner loop defines the distance between a pair of selection ports
of different regular stations.
In the symmetric
star topology, this distance is equal twice the distance of a regular
station from the {\tt Hub}.

\section{Traffic definition}

The simple traffic in our network is described in function {\tt initTraffic}
by one traffic pattern.
At the beginning, the function reads traffic-related protocol parameters from
the data file.
{\tt MinPL} and {\tt MaxPL} are the minimum and the maximum packet length,
respectively.
{\tt FrameL} stands for the length of the frame information, i.e., the
packet header and trailer.
These parameters are used by {\tt getPacket} to build packets from messages.
The next two values read from the input data file represent the mean
message inter-arrival time and the mean message length.

{\tt EchoTimeout} gives the amount of time that elapses until the
transmitter of a packet assumes that the packet has not made it through the
{\tt Hub}, if the packet echo does not appear in the meantime on the broadcast
port (see above).

{\tt HdrRecTime} denotes the time needed to recognize the packet's sender.
A station awaiting the echo of its packet must wait for {\tt HdrRecTime}
since the moment it detects the beginning of a packet, until it can definitely
tell whether the packet was transmitted by the station.

All regular stations (numbered from $1$ to {\tt NUsers}) are defined as
legitimate senders and receivers for the only traffic pattern created in
{\tt initTraffic}.
Their weights are identical and the traffic is uniform with respect to the
regular stations.

\section{The protocol code}

The last element of the startup action performed by {\tt Root} is the creation
of the protocol processes.

With the first loop, three processes are created for each regular station.
The argument in parentheses directly following the {\tt create} keyword
tells the {\tt Id} of the station which is to own the created process.
Note that regular stations are numbered from $1$ to {\tt NUsers}.

The {\tt Hub} runs {\tt NUsers} versions of the same {\tt HubProcess}.
The {\tt create} argument following the process type keyword is a character
string representing the process' ``nickname".
This nickname includes the numerical identifier of the station whose selection
link is serviced by the process.
This identifier is by one greater than the corresponding number of the
{\tt Hub}'s selection port (specified as the setup argument in the last pair
of parentheses).

\subsection{The {\tt Hub} process}

The goal of a {\tt Hub} process is to monitor its dedicated selection
link for incoming packets.

When started, the {\tt Hub} process awaits a packet arrival on its
selection port (event {\tt BOT}).
If the packet arrives while the broadcast medium is busy (the {\tt Hub} is
already relaying a packet from another selection port), the packet
is ignored (the {\tt BOT} event is {\em skipped\/}) and the process
awaits another packet arrival.
If the broadcast channel is idle, it is immediately marked as busy and the
packet is re-transmitted on the {\tt Hub}'s broadcast port.
When the transfer is complete, the {\tt Hub} status is reset to idle and the
process resumes waiting for another packet to relay.

\subsection{The transmitter}

The transmitter ({\tt Xmitter}) of a regular station cooperates with the
{\tt Monitor}.
Having acquired a packet from the client, the process sends a signal to the
{\tt Monitor} (by putting a dummy item into the {\tt StartEW} mailbox)
and starts to transmit the packet on the station's selection port.
At the same time, the process issues a wait request to the {\tt NACK}
mailbox for a possible ``negative acknowledgement" message coming from the
{\tt Monitor}.
If the packet is sufficiently long, it may happen that the
{\tt Monitor} decides to send the {\tt NACK} message before the
packet is entirely transmitted.
In such case, there is no point in continuing the transmission and the
process aborts the packet.
Note that nothing will be lost if the ``positive acknowledgement"
signal arrives from the {\tt Monitor} before the transfer is complete.
The signal will simply remain pending at the station
and it will be immediately accepted when the
transmitter issues a wait request
for it at state {\tt Done}.
All mailboxes (in particular {\tt ACK}) have been created with capacity
$1$ (see the {\tt User setup} method) which makes them capable of storing
(at most) one pending item.

\subsection{The monitor}

The {\tt Monitor} is implemented with the help of an auxiliary process which
is created and killed by the {\tt Monitor} dynamically.
To understand the rationale of such a solution let us discuss the
expected behavior of the {\tt Monitor}.

The process has nothing to do until it is awakened by the
{\tt StartEW} signal from the transmitter.
Then, it has to set up an alarm clock for {\tt EchoTimeout} time units.
While waiting for the timer to go off, the {\tt Monitor} is supposed to
listen to the broadcast port.
Whenever a packet arrives there, the process must wake up, examine
the packet header to identify the sender, and, if the sender happens to
be different from the process' station, the waiting must continue until
either the right packet arrives or the alarm clock goes off.
One natural way to implement the alarm clock is to issue a
simple timer request.
Let us note, however, that whenever the {\tt Monitor} is interrupted by a
packet
on the broadcast port and finds out that the packet has been sent
by some other station, it
will have to set the alarm clock again for an appropriately updated time
interval.
Thus, the process will have to keep track for how much time it has been
waiting so far and how much time still remains to wait.

It seems to be a somewhat simpler approach to create a special process to
implement an independent alarm clock that can wait for the requested amount
of time without unsolicited interruptions.
This is the role of the {\tt Trumpet} process.

When the {\tt Trumpet} process is created
it issues a wait request to the timer for {\tt Delay} (initialized to
{\tt EchoTimeout} by its {\tt setup} function) time units.
When that delay elapses, the process is restarted at {\tt Play}
where it sends a signal to its father and terminates.

Upon reception of the starting signal from the transmitter (via the
{\tt StartEW} mailbox),
the {\tt Monitor} creates an instance
of {\tt Trumpet} (pointed to by {\tt AClock}).
From now on, it will be examining
all packets appearing on the station's broadcast port
until it either gets a signal from {\tt Trumpet} (indicating that the
echo waiting timeout has elapsed)
or finds a packet sent by the current station.
Note that the {\tt Monitor} simulates the operation of recognizing the
packet's sender.
We assume that the sender can be determined {\tt HdrRecTime} time
units since the moment when the beginning of the packet was heard.
Intentionally, this delay corresponds to receiving (a part of) the
packet header.

If the awaited packet echo arrives before the alarm clock goes off,
the {\tt Trumpet} process pointed to by
{\tt AClock} is killed, but only if the timeout signal
is not already pending at the station.
The pending signal means that {\tt Trumpet} has already killed
itself: the two events have occurred at the same {\em ITU}.

\subsection{The receiver}

The {\tt Receiver} process (run by a regular station)
is completely independent of the other processes:
its sole purpose is to listen to the broadcast port and detect packets
addressed to its owner.

When started for the first time, the process
issues a {\tt wait} request to the broadcast port specifying that it wants
to be restarted by the earliest {\tt EMP} event---the end of a packet
addressed to its station.
This corresponds to the complete reception of a packet and is the only
interesting event from the viewpoint of the {\tt Receiver}.
Upon reception of a packet, the process calls the {\tt Client}'s method
{\tt receive} that updates some performance measures.

Having accepted the packet, the process is ready to await the arrival of the
next one.
However, before branching back to {\tt Wait}, the {\tt Receiver}
has to make sure
that the last {\tt EMP} event disappears from the port.
This is the reason for using {\tt skipto} instead of {\tt proceed}.
Otherwise, the process would loop infinitely on the same event:
the modeled time does not flow automatically
while a process is running.

\section{Comments on the mailbox semantics}

There are some subtleties associated with the semantics of mailbox
operations which can be briefly discussed in the context of this example.
All the mailboxes used in our {\em Hubnet\/} implementation are created
with the {\tt setup} argument $1$ which means that each of them can store
one pending item.
This seems natural in the case of {\tt ACK}.
Note that the {\tt Monitor} can send the {\tt ACK} message to the {\tt Xmitter}
before the latter issues the wait request to the {\tt ACK} mailbox.
This happens when the transmitted packet is longer than the round-trip
propagation delay between the {\tt Hub} and the transmitting station.
Such pending {\tt ACK} message cannot be lost, but is must remain
stored to restart the {\tt Xmitter} immediately when it issues the wait
request to {\tt ACK} at state {\tt Done}.

The situation is less clear in the case of the remaining mailboxes.
For instance, it seems at first sight that {\tt StartEW} can be a
capacity $0$ mailbox, i.e., it can be assumed that the {\tt Monitor} is
always waiting for {\tt NEWITEM} on this mailbox when the
{\tt Xmitter} executes the {\tt put} operation st state {\tt Retransmit}.
Unfortunately, it need not be the case.
Having sent one of the two signals ({\tt ACK}, {\tt NACK}) awaited by the
{\tt Xmitter}, the {\tt Monitor} executes {\tt proceed} to return to
state {\tt WaitSignal} where it will issue a wait request to
{\tt StartEW}.
The {\tt proceed} operation guarantees that the {\tt Monitor} will get
to {\tt WaitSignal} within a single {\em ITU}.
However, several other things can happen within the same {\em ITU\/} before
the {\tt Monitor} actually resumes its operation at {\tt StartEW}.
For example, the {\tt Xmitter} can be awakened (due to the signal sent by
the {\tt Monitor}), get to {\tt NewPacket}, acquire a new
packet for transmission, move to {\tt Retransmit} and execute the {\tt put}
operation for {\tt StartEW}.
All these actions will be performed within a single {\em ITU}, if only
there is a packet already waiting at the station.
In particular, it is possible that the {\tt put} operation at state
{\tt Retransmit} will be performed before the {\tt Monitor} has issued
a wait request for {\tt StartEW}.
Thus, if {\tt StartEW} is a capacity $0$ mailbox, the signal sent by the
{\tt Xmitter} will be lost.
Similar scenarios can be thought of for the remaining mailbox, i.e.,
{\tt NACK}.

Note that the event awaited on the mailboxes is {\tt RECEIVE}.
The semantics of this event is as follows:
\begin{itemize}
\item
If the mailbox is nonempty, the event occurs immediately, i.e., within the
current {\em ITU}.
\item
If the mailbox is empty, the event will occur at the first {\em ITU} when
the mailbox becomes nonempty.
\end{itemize}

Moreover, when the event is actually triggered and wakes up a process, the
item triggering the event is removed ({\tt RECEIVE}d) from the mailbox.
Thus, in our case of a capacity $1$ mailbox, the mailbox will appear empty
for a subsequent wait operation, until something is put into it.

There are some other events that can be awaited on a mailbox.
In particular, the semantics of {\tt NONEMPTY} is similar to the semantics
of {\tt RECEIVE}, except that the item triggering the event remains in
the mailbox and must be removed from there explicitly (by {\tt get} or
{\tt erase}).
Moreover, if multiple processes await the {\tt NONEMPTY} event on the same
mailbox and one item is put into the mailbox, all the waiting processes will
be restarted.
Thus, a {\tt get} operation issued after a restart due to the
{\tt NONEMPTY} event need not be successful, i.e., the mailbox may actually
appear empty.
This is impossible for {\tt RECEIVE}, i.e., in the above scenario only one
of the processes waiting for {\tt RECEIVE} will be awakened and will actually
get the queued item.

Yet another event that can be awaited on a mailbox is {\tt NEWITEM}.
This event is only triggered when something is actually
put into the mailbox.
For example, by issuing a wait request for {\tt NEWITEM} on a nonempty
mailbox the process will block itself until a {\bf new item} is put into
the mailbox.
No {\tt NONEMPTY} or {\tt RECEIVE}
event ever occurs on a capacity $0$ mailbox (which, by
definition, is always empty and nothing tangible
can be received from it), whereas
{\tt NEWITEM} is triggered for such a mailbox with each put operation.

\end{document}

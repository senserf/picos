\documentstyle [amssymbols,11pt] {article}
\input {../../../MANUAL/rm_en}

\begin{document}

\begin{titlepage}

\vspace*{3.5in}
\begin{center}
\Huge\bf TOKEN RING \vspace{0.25in}
\end{center}
\end{titlepage}

\section{The protocol}

In a ring-type network, stations are connected into a circular structure
in such a way that each station is directly attached to its two immediate
neighbors via two separate ports.
With our simple version of the {\em Token Ring\/} protocol,
information is passed along
the ring in one direction, e.g.\ 
clockwise.
Thus, each station receives information on one
(input) port and transmits on the other (output) port.
A special packet (token) is circulated among the stations.
If the station receiving the token has
no data packet awaiting transmission, it immediately relays the
token by copying it to the output port.
Otherwise, the station stops the token and transmits the
data packet instead.
Having completed the transmission of the data packet,
the station passes the token.

A
station may only transmit one data packet per each acquisition of the token:
the station is supposed to pass the token after transmitting at most one data
packet.
If a data packet arrives that is addressed to the current station,
it is received and removed from the ring.
Other data packets are passed around until eventually they
reach their proper destinations.

\section{Topology definition}

The network topology is defined by function {\tt genTopology} called from the
{\tt Root} process.
The function reads three parameters from the data file:
the number of links which is
equal to the number of stations (variables {\tt NChannels} and
{\tt NNodes}), the length of a link in {\em ITU\/}s
(we assume that all links are of the same length), and the transmission
rate of the ring.
The transmission rate tells the number of {\em ITU\/}s needed to insert a
single bit into the network.

Next, in a loop executed for all links, the function creates the links and the
stations.
A ring station is represented by an object of type {\tt Node}.
This station type declares two port pointers ({\tt IPort} and {\tt OPort}),
two packet buffers ({\tt RBuffer} and {\tt PBuffer}), and two
mailbox pointers ({\tt Xmit} and {\tt Pass}).
The two ports and the mailboxes
are created by the {\tt Node}'s {\tt setup} method.
{\tt IPort} is the receiving (input)
port and the other port is the transmitting (output) port.
The transmission rate of both ports is
set to {\tt TRate} which is read from the input file.
{\tt RBuffer} is used to store packets that arrive on the input port.
Such a packet has to be either
received (if it is a regular packet addressed to this
station) or relayed to the next station in the ring.
The other packet buffer stores packets that originate at the station (are
acquired by the station from the {\tt Client}).
The two mailboxes are used to synchronize two processes run at each station.

The second loop of function {\tt genTopology} goes through all links, connects
them to the stations, and assigns lengths to them.
A link number {\tt i} (by the link number we mean the contents of its {\tt Id}
attribute) connects stations number {\tt i} and
{\tt i}$+$1{\tt ~mod~NNodes},
or, more specifically, {\tt OPort} of station {\tt i} to {\tt IPort} of
station
{\tt i}$+$1{\tt ~mod~NNodes}.

\section{Traffic}

The traffic in the ring consists of two types of packets: regular packets
and the token.
Client messages and regular packets built from them are represented by the
standard types {\tt Message} and {\tt Packet};
the token is a special packet of type {\tt Token}.
The (total) length of the token is determined by the
input parameter ({\tt TokL})
read by the function {\tt genTraffic} which takes care of initializing the
client operation.
It creates a single traffic pattern describing the message arrival
process (exponentially distributed message length and inter-arrival time with
the mean values read from the data file).

Five more input values determine the packet framing.
{\tt MinPL}, {\tt MaxPL}, and {\tt FrmL} stand for the minimum packet length,
the maximum packet length, and the combined length of the packet header
and trailer.
These parameters apply to regular packets that will be acquired for transmission
from the {\tt Client}.
{\tt HdrL} is the length of an initial part of the packet header that must
be read by a station
in order to recognize the packet type and its destination
(see the next section).
Finally, {\tt TokL} is the already mentioned total length of the token packet.
All these values are in bits.

\section{Protocol implementation}

A natural way of implementing the protocol is to have two
processes at each station, one process (of type {\tt IProcess})
servicing the input port ({\tt IPort}) and processing
packets arriving at the station, the other process (of type {\tt OProcess})
handling the output port ({\tt OPort})
and passing packets to the next station on the ring.
The two processes communicate via signals passed through mailboxes
{\tt Xmit} and {\tt Pass}.

The {\tt Root}'s method {\tt genProtocol} creates the two processes for each
station.
Note the special syntax of {\tt create}.
The value of {\tt i} (in parentheses) identifies the station in the context
of which the process is to be created.

\subsection{The input process}

Upon setup, the input process creates its private pointers to the station's
input port ({\tt IPort}) and the relay buffer ({\tt RBuffer}).
Then, if the station {\tt Id} is $0$, the process creates the token packet
and stores it in the relay buffer.
Thus station $0$ is responsible for the protocol initialization.
A signal is sent to the output process (by storing a
dummy item in the {\tt Xmit} mailbox) to notify it that
station $0$ is in possession of the token (see the next subsection).
A similar action is performed by {\tt IProcess} whenever the token is received
on the input port.

The ``main loop" of {\tt IProcess} code starts at state
{\tt WaitPacket} where the process awaits a packet arrival on the
input port.
As soon as the beginning of a packet is detected, the packet is put into
{\tt RBuffer} and then the process simulates the operation of
recognizing (an initial part of) the packet's header.
Before some part of the header
is received, the input process cannot know what to do with the packet.
Thus the further processing of the packet is delayed by the time
interval corresponding to receiving {\tt HdrL} bits from {\tt IPort}.
Note that although {\tt IPort} is never used for transmitting packets,
its transmission rate is relevant: it is used to determine the delay
needed to recognize the interesting part of the packet header.
Of course, in this particular case, as transmission rates of all ports are
identical, {\tt OPort} could be used to determine this delay.

At state {\tt Recognized} the process determines the type of the
packet.
If it is the token, the processing continues at state {\tt GotToken} where
a signal is sent to the output process to notify it that now the station
is allowed to transmit its own packet.
Then the input process resumes waiting for a packet on {\tt IPort}.

If the arriving packet is a data packet addressed to the
station, the process waits until the packet's end appears on the port and then
it {\tt receives} the packet.
A data packet addressed to some other station must be relayed by the
output process (essentially, in the same way as the token).
Thus, upon arrival of such a packet, the input process sends a special
signal (mailbox {\tt Pass}) to the output process.

\subsection{The output process}

The output process ({\tt OProcess}) sleeps until it receives
one of two signals from the input process.
A signal arriving on the {\tt Xmit} mailbox
indicates that the station has just
received the token and it is allowed to transmit its own data packet.
The output process attempts to acquire a packet from the {\tt Client}.
If this operation is successful,
the acquired packet (the contents of {\tt PBuffer}) is sent on the output port.
Then the process gets to state {\tt OwnDone} where it
{\tt releases} the client packet and
transmits the token to the station's successor on the ring.

If there is no client packet awaiting transmission, the token (from
{\tt RBuffer}) is immediately relayed to the next station on the ring.
The same happens to a data packet addressed to another station (after
receiving a signal on the {\tt Pass} mailbox): the processing of such a packet
is identical to passing the token.

\end{document}

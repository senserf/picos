\section{Programming Protocols in \smurph}

We assume that the user is well acquainted with C++ and with the terminology
used in object-oriented programming.

The user-supplied type and data definitions together with the code of the
protocol processes are contained in a C++ file (or a number of C++ files).
They look like a regular program in C++ which has access to the \smurph\
libraries of types, data structures, functions, and macro-operations.
A special support program ({\tt mks}) is provided whose purpose is to
merge the user protocol files with the \smurph\ libraries and create
a stand-alone version of the simulator.

\subsection{Time}

Time in \smurph\ is discrete which means that there is an
{\em indivisible time unit\/} ({\em ITU\/}) and two events occurring during
the same {\em ITU\/} are not ordered deterministically with respect to their
actual succession.
The indivisible time unit may correspond to an
arbitrary time interval in the modeled physical system.
Time intervals are represented as objects of type {\tt TIME}.
The precision of {\tt TIME} is selected by the user.
There is no explicit limit on this precision;
thus, the {\em ITU\/} can correspond to a very small unit of real
time\footnote{Theoretically, one could even get down to the Planck time (about
$10^{-45}$s). Thus, the discrete nature of protocol modeling in {\em Smurph\/}
 is not really a limitation.}.
The only penalty for using a very high resolution of {\tt TIME} is the
reduced (real) execution speed.
Standard arithmetic operations on objects of type {\tt TIME} and combinations
of these objects with other numeric entities are available.
They are implemented by overloading the usual arithmetic operators:
the user does not have to handle {\tt TIME} objects differently from other
numbers.

Besides the {\em ITU}, \smurph\ defines another unit of time, the so-called
{\em ETU\/} which stands for the {\em experimenter time unit}.
This unit is used for presenting the simulation results.

\subsection{Hierarchy of compound types}

All user-visible compound types\footnote{From now on, the words
{\em type\/} and {\em class\/} will be used interchangeably---to denote the
same thing.} (we conveniently assume that
{\tt TIME} is a simple type) are organized into a hierarchy presented on
\fig{hot}.
This hierarchy reflects the inclusion of types in terms of the C++ subclass
concept.

\begin{figure}[htbp]
	\begin{center}
		\catcode`@=11
\expandafter\ifx\csname graph\endcsname\relax \alloc@4\box\chardef\insc@unt\graph\fi
\catcode`@=12
\setbox\graph=\vtop{%
  \baselineskip=0pt \lineskip=0pt \lineskiplimit=0pt
  \vbox to0pt{\hbox{%
    \special{pn 8}%
    \special{ar 1775 50 2 2  0.000  6.283}%
    \special{ar 1775 50 5 5  0.000  6.283}%
    \special{ar 1775 50 7 7  0.000  6.283}%
    \special{ar 1775 50 10 10  0.000  6.283}%
    \special{ar 1775 50 13 13  0.000  6.283}%
    \special{ar 1775 50 15 15  0.000  6.283}%
    \special{ar 1775 50 18 18  0.000  6.283}%
    \special{ar 1775 50 20 20  0.000  6.283}%
    \special{ar 1775 50 23 23  0.000  6.283}%
    \special{ar 1775 50 25 25  0.000  6.283}%
    \rlap{\kern  1.775in\lower-0.041in\hbox to 0pt{\hss {\small {\em class}}\hss}}%
    \special{pa 1775 50}%
    \special{pa 1150 1425}%
    \special{fp}%
    \special{ar 1150 1425 2 2  0.000  6.283}%
    \special{ar 1150 1425 5 5  0.000  6.283}%
    \special{ar 1150 1425 7 7  0.000  6.283}%
    \special{ar 1150 1425 10 10  0.000  6.283}%
    \special{ar 1150 1425 13 13  0.000  6.283}%
    \special{ar 1150 1425 15 15  0.000  6.283}%
    \special{ar 1150 1425 18 18  0.000  6.283}%
    \special{ar 1150 1425 20 20  0.000  6.283}%
    \special{ar 1150 1425 23 23  0.000  6.283}%
    \special{ar 1150 1425 25 25  0.000  6.283}%
    \rlap{\kern  1.100in\lower 1.416in\hbox to 0pt{\hss {\small {\em Object}}}}%
    \special{pa 1775 50}%
    \special{pa 2025 1550}%
    \special{fp}%
    \special{ar 2025 1550 2 2  0.000  6.283}%
    \special{ar 2025 1550 5 5  0.000  6.283}%
    \special{ar 2025 1550 7 7  0.000  6.283}%
    \special{ar 2025 1550 10 10  0.000  6.283}%
    \special{ar 2025 1550 13 13  0.000  6.283}%
    \special{ar 2025 1550 15 15  0.000  6.283}%
    \special{ar 2025 1550 18 18  0.000  6.283}%
    \special{ar 2025 1550 20 20  0.000  6.283}%
    \special{ar 2025 1550 23 23  0.000  6.283}%
    \special{ar 2025 1550 25 25  0.000  6.283}%
    \rlap{\kern  2.025in\lower 1.724in\hbox to 0pt{\hss {\small {\tt Message}}\hss}}%
    \special{pa 1775 50}%
    \special{pa 2650 1550}%
    \special{fp}%
    \special{ar 2650 1550 2 2  0.000  6.283}%
    \special{ar 2650 1550 5 5  0.000  6.283}%
    \special{ar 2650 1550 7 7  0.000  6.283}%
    \special{ar 2650 1550 10 10  0.000  6.283}%
    \special{ar 2650 1550 13 13  0.000  6.283}%
    \special{ar 2650 1550 15 15  0.000  6.283}%
    \special{ar 2650 1550 18 18  0.000  6.283}%
    \special{ar 2650 1550 20 20  0.000  6.283}%
    \special{ar 2650 1550 23 23  0.000  6.283}%
    \special{ar 2650 1550 25 25  0.000  6.283}%
    \rlap{\kern  2.650in\lower 1.724in\hbox to 0pt{\hss {\small {\tt Packet}}\hss}}%
    \special{pa 1775 50}%
    \special{pa 3150 1050}%
    \special{fp}%
    \special{ar 3150 1050 2 2  0.000  6.283}%
    \special{ar 3150 1050 5 5  0.000  6.283}%
    \special{ar 3150 1050 7 7  0.000  6.283}%
    \special{ar 3150 1050 10 10  0.000  6.283}%
    \special{ar 3150 1050 13 13  0.000  6.283}%
    \special{ar 3150 1050 15 15  0.000  6.283}%
    \special{ar 3150 1050 18 18  0.000  6.283}%
    \special{ar 3150 1050 20 20  0.000  6.283}%
    \special{ar 3150 1050 23 23  0.000  6.283}%
    \special{ar 3150 1050 25 25  0.000  6.283}%
    \rlap{\kern  3.150in\lower 1.224in\hbox to 0pt{\hss {\small {\tt SGroup}}\hss}}%
    \special{pa 1775 50}%
    \special{pa 3275 550}%
    \special{fp}%
    \special{ar 3275 550 2 2  0.000  6.283}%
    \special{ar 3275 550 5 5  0.000  6.283}%
    \special{ar 3275 550 7 7  0.000  6.283}%
    \special{ar 3275 550 10 10  0.000  6.283}%
    \special{ar 3275 550 13 13  0.000  6.283}%
    \special{ar 3275 550 15 15  0.000  6.283}%
    \special{ar 3275 550 18 18  0.000  6.283}%
    \special{ar 3275 550 20 20  0.000  6.283}%
    \special{ar 3275 550 23 23  0.000  6.283}%
    \special{ar 3275 550 25 25  0.000  6.283}%
    \rlap{\kern  3.275in\lower 0.724in\hbox to 0pt{\hss {\small {\tt CGroup}}\hss}}%
    \special{pa 1150 1425}%
    \special{pa 25 1800}%
    \special{fp}%
    \special{ar 25 1800 2 2  0.000  6.283}%
    \special{ar 25 1800 5 5  0.000  6.283}%
    \special{ar 25 1800 7 7  0.000  6.283}%
    \special{ar 25 1800 10 10  0.000  6.283}%
    \special{ar 25 1800 13 13  0.000  6.283}%
    \special{ar 25 1800 15 15  0.000  6.283}%
    \special{ar 25 1800 18 18  0.000  6.283}%
    \special{ar 25 1800 20 20  0.000  6.283}%
    \special{ar 25 1800 23 23  0.000  6.283}%
    \special{ar 25 1800 25 25  0.000  6.283}%
    \rlap{\kern  0.025in\lower 1.974in\hbox to 0pt{\hss {\small {\tt RVariable}}\hss}}%
    \special{pa 1150 1425}%
    \special{pa 775 1800}%
    \special{fp}%
    \special{ar 775 1800 2 2  0.000  6.283}%
    \special{ar 775 1800 5 5  0.000  6.283}%
    \special{ar 775 1800 7 7  0.000  6.283}%
    \special{ar 775 1800 10 10  0.000  6.283}%
    \special{ar 775 1800 13 13  0.000  6.283}%
    \special{ar 775 1800 15 15  0.000  6.283}%
    \special{ar 775 1800 18 18  0.000  6.283}%
    \special{ar 775 1800 20 20  0.000  6.283}%
    \special{ar 775 1800 23 23  0.000  6.283}%
    \special{ar 775 1800 25 25  0.000  6.283}%
    \rlap{\kern  0.775in\lower 1.974in\hbox to 0pt{\hss {\small {\tt Station}}\hss}}%
    \special{pa 1150 1425}%
    \special{pa 1900 2300}%
    \special{fp}%
    \special{ar 1900 2300 2 2  0.000  6.283}%
    \special{ar 1900 2300 5 5  0.000  6.283}%
    \special{ar 1900 2300 7 7  0.000  6.283}%
    \special{ar 1900 2300 10 10  0.000  6.283}%
    \special{ar 1900 2300 13 13  0.000  6.283}%
    \special{ar 1900 2300 15 15  0.000  6.283}%
    \special{ar 1900 2300 18 18  0.000  6.283}%
    \special{ar 1900 2300 20 20  0.000  6.283}%
    \special{ar 1900 2300 23 23  0.000  6.283}%
    \special{ar 1900 2300 25 25  0.000  6.283}%
    \rlap{\kern  1.950in\lower 2.291in\hbox to0pt{{\small {\tt AI}}\hss}}%
    \special{pa 1150 1425}%
    \special{pa 2150 2050}%
    \special{fp}%
    \special{ar 2150 2050 2 2  0.000  6.283}%
    \special{ar 2150 2050 5 5  0.000  6.283}%
    \special{ar 2150 2050 7 7  0.000  6.283}%
    \special{ar 2150 2050 10 10  0.000  6.283}%
    \special{ar 2150 2050 13 13  0.000  6.283}%
    \special{ar 2150 2050 15 15  0.000  6.283}%
    \special{ar 2150 2050 18 18  0.000  6.283}%
    \special{ar 2150 2050 20 20  0.000  6.283}%
    \special{ar 2150 2050 23 23  0.000  6.283}%
    \special{ar 2150 2050 25 25  0.000  6.283}%
    \rlap{\kern  2.200in\lower 2.041in\hbox to0pt{{\small {\tt EObject}}\hss}}%
    \special{pa 1900 2300}%
    \special{pa 525 2800}%
    \special{fp}%
    \special{ar 525 2800 2 2  0.000  6.283}%
    \special{ar 525 2800 5 5  0.000  6.283}%
    \special{ar 525 2800 7 7  0.000  6.283}%
    \special{ar 525 2800 10 10  0.000  6.283}%
    \special{ar 525 2800 13 13  0.000  6.283}%
    \special{ar 525 2800 15 15  0.000  6.283}%
    \special{ar 525 2800 18 18  0.000  6.283}%
    \special{ar 525 2800 20 20  0.000  6.283}%
    \special{ar 525 2800 23 23  0.000  6.283}%
    \special{ar 525 2800 25 25  0.000  6.283}%
    \rlap{\kern  0.525in\lower 2.974in\hbox to 0pt{\hss {\small {\tt Timer}}\hss}}%
    \special{pa 1900 2300}%
    \special{pa 900 3050}%
    \special{fp}%
    \special{ar 900 3050 2 2  0.000  6.283}%
    \special{ar 900 3050 5 5  0.000  6.283}%
    \special{ar 900 3050 7 7  0.000  6.283}%
    \special{ar 900 3050 10 10  0.000  6.283}%
    \special{ar 900 3050 13 13  0.000  6.283}%
    \special{ar 900 3050 15 15  0.000  6.283}%
    \special{ar 900 3050 18 18  0.000  6.283}%
    \special{ar 900 3050 20 20  0.000  6.283}%
    \special{ar 900 3050 23 23  0.000  6.283}%
    \special{ar 900 3050 25 25  0.000  6.283}%
    \rlap{\kern  0.900in\lower 3.224in\hbox to 0pt{\hss {\small {\tt Client}}\hss}}%
    \special{pa 1900 2300}%
    \special{pa 1275 3300}%
    \special{fp}%
    \special{ar 1275 3300 2 2  0.000  6.283}%
    \special{ar 1275 3300 5 5  0.000  6.283}%
    \special{ar 1275 3300 7 7  0.000  6.283}%
    \special{ar 1275 3300 10 10  0.000  6.283}%
    \special{ar 1275 3300 13 13  0.000  6.283}%
    \special{ar 1275 3300 15 15  0.000  6.283}%
    \special{ar 1275 3300 18 18  0.000  6.283}%
    \special{ar 1275 3300 20 20  0.000  6.283}%
    \special{ar 1275 3300 23 23  0.000  6.283}%
    \special{ar 1275 3300 25 25  0.000  6.283}%
    \rlap{\kern  1.275in\lower 3.474in\hbox to 0pt{\hss {\small {\tt Traffic}}\hss}}%
    \special{pa 1900 2300}%
    \special{pa 2025 3550}%
    \special{fp}%
    \special{ar 2025 3550 2 2  0.000  6.283}%
    \special{ar 2025 3550 5 5  0.000  6.283}%
    \special{ar 2025 3550 7 7  0.000  6.283}%
    \special{ar 2025 3550 10 10  0.000  6.283}%
    \special{ar 2025 3550 13 13  0.000  6.283}%
    \special{ar 2025 3550 15 15  0.000  6.283}%
    \special{ar 2025 3550 18 18  0.000  6.283}%
    \special{ar 2025 3550 20 20  0.000  6.283}%
    \special{ar 2025 3550 23 23  0.000  6.283}%
    \special{ar 2025 3550 25 25  0.000  6.283}%
    \rlap{\kern  2.025in\lower 3.724in\hbox to 0pt{\hss {\small {\tt Link}}\hss}}%
    \special{pa 1900 2300}%
    \special{pa 2400 3300}%
    \special{fp}%
    \special{ar 2400 3300 2 2  0.000  6.283}%
    \special{ar 2400 3300 5 5  0.000  6.283}%
    \special{ar 2400 3300 7 7  0.000  6.283}%
    \special{ar 2400 3300 10 10  0.000  6.283}%
    \special{ar 2400 3300 13 13  0.000  6.283}%
    \special{ar 2400 3300 15 15  0.000  6.283}%
    \special{ar 2400 3300 18 18  0.000  6.283}%
    \special{ar 2400 3300 20 20  0.000  6.283}%
    \special{ar 2400 3300 23 23  0.000  6.283}%
    \special{ar 2400 3300 25 25  0.000  6.283}%
    \rlap{\kern  2.400in\lower 3.474in\hbox to 0pt{\hss {\small {\tt Mailbox}}\hss}}%
    \special{pa 1900 2300}%
    \special{pa 2775 3050}%
    \special{fp}%
    \special{ar 2775 3050 2 2  0.000  6.283}%
    \special{ar 2775 3050 5 5  0.000  6.283}%
    \special{ar 2775 3050 7 7  0.000  6.283}%
    \special{ar 2775 3050 10 10  0.000  6.283}%
    \special{ar 2775 3050 13 13  0.000  6.283}%
    \special{ar 2775 3050 15 15  0.000  6.283}%
    \special{ar 2775 3050 18 18  0.000  6.283}%
    \special{ar 2775 3050 20 20  0.000  6.283}%
    \special{ar 2775 3050 23 23  0.000  6.283}%
    \special{ar 2775 3050 25 25  0.000  6.283}%
    \rlap{\kern  2.775in\lower 3.224in\hbox to 0pt{\hss {\small {\tt Port}}\hss}}%
    \special{pa 1900 2300}%
    \special{pa 3150 2800}%
    \special{fp}%
    \special{ar 3150 2800 2 2  0.000  6.283}%
    \special{ar 3150 2800 5 5  0.000  6.283}%
    \special{ar 3150 2800 7 7  0.000  6.283}%
    \special{ar 3150 2800 10 10  0.000  6.283}%
    \special{ar 3150 2800 13 13  0.000  6.283}%
    \special{ar 3150 2800 15 15  0.000  6.283}%
    \special{ar 3150 2800 18 18  0.000  6.283}%
    \special{ar 3150 2800 20 20  0.000  6.283}%
    \special{ar 3150 2800 23 23  0.000  6.283}%
    \special{ar 3150 2800 25 25  0.000  6.283}%
    \rlap{\kern  3.150in\lower 2.974in\hbox to 0pt{\hss {\small {\tt Process}}\hss}}%
    \special{pa 1900 2300}%
    \special{pa 3525 2550}%
    \special{fp}%
    \special{ar 3525 2550 2 2  0.000  6.283}%
    \special{ar 3525 2550 5 5  0.000  6.283}%
    \special{ar 3525 2550 7 7  0.000  6.283}%
    \special{ar 3525 2550 10 10  0.000  6.283}%
    \special{ar 3525 2550 13 13  0.000  6.283}%
    \special{ar 3525 2550 15 15  0.000  6.283}%
    \special{ar 3525 2550 18 18  0.000  6.283}%
    \special{ar 3525 2550 20 20  0.000  6.283}%
    \special{ar 3525 2550 23 23  0.000  6.283}%
    \special{ar 3525 2550 25 25  0.000  6.283}%
    \rlap{\kern  3.525in\lower 2.724in\hbox to 0pt{\hss {\small {\tt Observer}}\hss}}%
    \kern  3.551in
  }\vss}%
  \kern  3.725in
}

		\usebox{\graph}
		\vspace{0.75cm}
\caption{ The hierarchy of user-visible compound types.  \label{fig.hot}}
	\end{center}
\end{figure}

All compound objects in \smurph\ are instances of some classes.
This fact is represented by the root of the tree on \fig{hot}.
All objects exhibiting dynamic behavior belong to type {\tt Object}.
This type declares a number of standard attributes and methods
that each {\tt Object} must have.
Among them are: numeric and textual identifiers
(used for printing and displaying information about objects), 
the object qualification which determines the actual type of the object,
and a virtual function describing the way the object is to be {\em exposed}.
By {\em exposing\/} an object, we mean printing some information related
to the object or displaying this information in a
terminal window\footnote{This part is done by a separate program called
{\em DSD}.}.

Most of the standard subtypes of type {\tt Object} can (and sometimes must)
be extended by the user.
For example, the type {\tt Station} defines a skeleton for creating
{\em stations\/}---the network processing units (nodes) that run the protocol.
Typically, this skeleton has to be augmented by some protocol-specific
elements before it can be used to create actual stations.

There is a special category of dynamic objects called
{\em activity interpreters}, or {\em AI\/}s, for short.
These objects provide the protocol's interface with the outer world.
An {\em AI\/} can be viewed as a {\em daemon\/} that absorbs the protocol
{\em activities\/} and turns them into future events.

Type {\tt EObject} is the base class for defining non-standard
exposable object types.

Types of objects that do not act on their own,
but instead are subjects of activities of other objects,
are straightforward classes.
An example of such a type is {\tt Packet} describing a packet skeleton.
No packet ever performs any operation: packets are passive
objects which are handled by active objects.

\subsection{Type declaration}

The standard declaration apparatus of C++ (which can be naturally used in the
user-supplied protocol program) is extended in \smurph\ by a collection of
operations that are used to define extensions of certain
standard types from \fig{hot}.
All these operations obey a common philosophy:
the declaration of a \smurph\ type extension has the following form:
{\small
\begin{verbatim}
        keyword typename : suptypename {
            . . . .
            . . . .
        };
\end{verbatim} }
\noindent
where {\tt keyword} identifies a category of types
describing a common superclass of objects, e.g.:
processes, stations, traffic patterns;
{\tt typename} is the name of the defined type (class); and {\tt suptypename} is
the identifier of the immediate supertype (superclass) in which the new
type is contained.
This last part need not occur if the defined type is derived directly from
the corresponding base type (\fig{hot}).
The opening brace ({\tt \{}) is sometimes preceded by additional arguments
encapsulated in parentheses.
It is also possible to define new types as combinations of already
defined types, taking advantage of the {\em multiple inheritance\/}
apparatus of C++ (in such case {\tt suptypename} is a list of type names).

The declaration body (the text between the braces)
contains declarations of new attributes and methods.
From the formal point of view, the effect of a type extension operation is
equivalent to the following construct of C++:
{\small
\begin{tabbing}
{\tt 12345678} \= {\tt 1234} \= {\tt 1234} \kill
\> {\tt class typename : public suptypename \{ }\\
\> \> \ldots {\em standard prelude inserted by the macro-operation} \ldots \\
\> \> {\tt public:} \\
\> \> \ldots {\em user-defined attributes and methods} \ldots \\
\> {\tt \};}
\end{tabbing} }

If the type extension declares a method called {\tt setup},
this method will be automatically called when an object of the extended type is
created.
The {\tt setup} method plays the role of the object
{\em constructor\/}\footnote{Standard argument-less
constructors of C++ can also be used:
the effect of the standard constructor is combined with the effect of
{\tt setup}.}.

Class types declared by means of the standard type extension macro-operations
can be used to generate objects belonging to these types.
Again, \smurph\ provides its private tool for generating such objects.
The following operation:
{\small
\begin{verbatim}
        create typename ( ... )
\end{verbatim} }
\noindent
creates an object of type {\tt typename}.
Operation {\tt create} can be viewed as
a function returning the pointer to the created object.
The contents of the second pair of parentheses are passed as arguments to
the object's {\tt setup} method (the parentheses can be skipped if the method
is undefined or if it takes no arguments).
Deallocation of objects is performed by the standard C++ operation {\tt delete}.

\subsection{Describing network topology}

The topology of the modeled network is defined as interconnection of
{\em stations}, {\em ports}, and {\em links}.
Stations are objects belonging to type {\tt Station}; subtypes of this type
can be defined by the user to describe stations with specific characteristics.
Stations represent dynamic elements of the network.
The protocol is described as a collection of processes (objects of type
{\tt Process}) which are executed by stations.

A station is typically attached to a link (or a number of links) via ports.
A port (an object of type {\tt Port}) represents a connection of one
station to one link.
A link (an object of type {\tt Link}) models a simple
communication medium, e.g.:
a (possibly unidirectional) fiber optic carrier, a coaxial cable,
or a radio channel.

For each pair of ports connecting some station(s) to the same link, the
distance between the ports is specified as the number of {\em ITU\/}s
needed to propagate a signal from one port to the other\footnote{For a
bidirectional link, the distance is the same in both directions, whereas for
a unidirectional link, only one-way distance is defined.}.
Thus, time is used to express distances in the network; the model
abstracts from the actual length and propagation speed of the media.

One attribute of a port is its {\em transmission rate\/} which says how
fast information can be inserted into the port.
This parameter is expressed as the amount of time (in {\em ITU\/}s)
required to ``pump" a single bit into the port.
Transmission rates of different ports
(even if they are connected to the same link) need not be the same.

Stations, ports, and links are assigned numeric identifiers which generally
correspond to the order in which the particular objects have been created.
Besides the numeric identifiers (which are always assigned), optional
textual names can be associated with
these objects---to be used for printing and/or displaying.

\subsubsection{Stations}

A network modeled in \smurph\ may include stations of several types.
For example, in a star-shaped network, the central node plays the
role of a switching device whose structure and behavior may be completely
different from those of a regular station.
The basic type for building stations is {\tt Station}.
A specific station type is declared in the following way:
{\small
\begin{tabbing}
{\tt 12345678} \= {\tt 1234} \= {\tt 1234} \kill
\> {\tt station stype : supstype \{} \\
\> \> \ldots {\em attributes and methods} \ldots \\
\> {\tt \};}
\end{tabbing} }
\noindent
where {\tt stype} is the declared type name and {\tt supstype} is a previously
declared station type ({\tt supstype}, together with the preceding colon,
can be skipped, if it is {\tt Station}---see \fig{hot}).

The attributes and methods of the new station type are combined with the
attributes and methods of the supertype---in the regular way determined by the
semantics of the C++ class declaration.
In particular, the new {\tt setup} method (if defined) subsumes the {\tt setup}
method defined in the supertype.

A station owns a number of standard attributes inherited from
{\tt Station} and {\tt Object}.
The attributes coming from {\tt Object} are mostly protected and hidden by
a number of standard methods and macro-operations.
The user-visible standard part of a station type consists of two arrays of
pointers:
{\small
\begin{verbatim}
        Message         **MQHead,
                        **MQTail;
\end{verbatim} }
\noindent
pointing to message queues, each message
queue holding messages belonging to a specific {\em traffic pattern}.
These queues are filled (in most cases automatically) by the programmable
traffic generator called the {\tt Client}.

An actual station object is created by invoking {\tt create} with the station
type name as the parameter, i.e.:
{\small
\begin{tabbing}
{\tt 12345678} \= {\tt 1234} \= {\tt 1234} \kill
\> {\tt st = create stype (} {\em setup arguments\/} {\tt );} \\
\end{tabbing} }
The second pair of parentheses contains arguments of the station's {\tt setup}
function.
Stations are assigned numeric identifiers (these identifiers are
attributes of {\tt Object}) in the order of their creation.
The first created stations is numbered $0$.
The station identifier is not directly visible to the user and it cannot be
changed.
Two standard operations:
{\small
\begin{verbatim}
        Station         *idToStation (int Id);
\end{verbatim} }
\noindent
and
{\small
\begin{verbatim}
        int             ident (Station *st);
\end{verbatim} }
\noindent
convert station identifiers to object pointers and vice versa.

\subsubsection{Links}

The predefined link types offered by \smurph\ are seldom extended in the
protocol program.
The only reasonable situation when such an extension could be warranted is
the
modification of the standard functions for calculating link-related
performance measures.
These functions, as well as other built-in performance calculating functions
of \smurph, are virtual, which makes them easily exchangeable by a simple type
extension.
\smurph\ provides the user with four standard link types which are in fact
two different implementations of two basic link concepts: the broadcast
bidirectional link (types {\tt BLink} and {\tt CLink}) and the unidirectional
link ({\tt ULink} and {\tt PLink}).
A specific link instance is generated by calling {\tt create} in the following
way:
{\small
\begin{verbatim}
        lk = create ltype (np, at, sp);
\end{verbatim} }
\noindent
where {\tt ltype} is one of the above-mentioned four link types.
The three arguments are passed to the standard link {\tt setup} function
and they have the following meaning:
\begin{llist}{{\tt nppp}}
\litem{{\tt np}}
The number of ports connected to the link.
\litem{{\tt at}}
The archival time limit.
Activities leaving a link are kept for some time in the so-called
link {\em archive}.
The link archive can be used by the protocol program to inquire the link
about its past status, down to the limit determined by the value of {\tt at}.
\litem{{\tt sp}}
A boolean ({\tt int}-type) flag that tells whether the link-related standard
performance measures are to be calculated.
This flag can assume one of two values: {\tt ON} or {\tt OFF}.
\end{llist}

Only the first argument in the second pair of parentheses is required.
The default values for the remaining two arguments are: {\tt 0} (archive
not maintained) and {\tt ON} (standard performance measures to be
calculated), respectively.

Similarly to stations, links are also numbered in the order of their
creation.
A link can be referenced either via a pointer to the link object or by its
numeric identifier.
One situation when the user may want to reference a link explicitly is
the definition of a port (see below) which must be assigned to a specific
link.

\subsubsection{Ports}

Type {\tt Port} is never extended by the user.
A port is created in the following way:
{\small
\begin{verbatim}
        pt = create Port (rate);
\end{verbatim} }
\noindent
where the argument (passed to the port's {\tt setup} method) gives
the port transmission rate---the number of {\em ITU\/}s required to insert a
single bit into the port.

To describe a network topology, ports, links, and stations must be organized
into a single interconnected structure.
A port is usually created in the context of a specific station, typically
within the station's {\tt setup} method.
Such a port belongs to the given station and
its purpose is to connect the station to a link.
The following operation connects the port pointed to by
{\tt pt} to the link pointed to by {\tt lk}:
{\small
\begin{verbatim}
        pt->connect (lk);
\end{verbatim} }

The geometry of a link is described by a {\em distance matrix\/} that specifies
distances between all pairs of ports connected to the link.
There are a number of ways to specify a link distance matrix; the most
natural way is to explicitly assign a distance to each combination of
two ports connected to the link.
The distance between a pair of ports can be assigned by calling:
{\small
\begin{verbatim}
        p1->setDTo (p2, d);
\end{verbatim} }
\noindent
where {\tt p1} and {\tt p2} are port pointers and {\tt d} is the distance
between the two ports.

\subsection{Defining traffic conditions}

The traffic in the network consists of {\em messages\/} which arrive from
``outside" to be processed by stations.
By processing a message, we mean splitting it into one or more packets
and transmitting it over the network to the destination (a specific station).
The traffic distribution is described by a collection of the so-called
{\em traffic patterns} which are objects of type {\tt Traffic}.
Each traffic pattern is associated with specific message and packet types.
Messages and packets belonging to different traffic patterns may have different,
protocol-dependent structures.

\subsubsection{Messages and packets}

A traffic pattern consists of a message inter-arrival process and two
types representing messages and packets to be
generated according to the pattern.
These types are defined on the basis of two standard classes {\tt Message} and
{\tt Packet}.
The built-in part of the message structure (inherited from {\tt Message})
has the following contents:
{\small
\begin{verbatim}
        int             Receiver,
                        TP;
        TIME            QTime;
        long            Length;
\end{verbatim} }
\noindent
where {\tt Receiver} is the identifier of the station that is to receive
the message (a special value is used to represent a broadcast message addressed
to a set of stations), {\tt TP} identifies the traffic pattern that was used to
generate the message, {\tt QTime} tells the time when the message was generated
and queued at the station (this attribute is used for measuring the
message delay),
and {\tt Length} gives the message length in bits.

The user-visible standard contents of a packet have the following layout:
{\small
\begin{verbatim}
        int             Sender,
                        Receiver,
                        TP;
        TIME            QTime,
                        TTime;
        long            ILength,
                        TLength,
                        Flags;
        int             isMy ();
\end{verbatim} }

Arguments {\tt Receiver}, {\tt TP}, and {\tt QTime} are direct copies of the
corresponding arguments from {\tt Message}.
{\tt Sender} identifies the packet's original
transmitter---the station at which the message was queued for
transmission.
{\tt TTime} (the so-called {\em top time\/}) contains the time when the packet
became ready for transmission (this attribute is used for calculating the
packet delay).
{\tt ILength} and {\tt TLength} specify the length of the information part
of the packet (the part acquired from the message) and the total packet length
(including the possible header and trailer), respectively.
{\tt Flags} is a collection of binary attributes of the packet; usually
they are not referenced directly.

The method {\tt isMy} can be used by a station to determine whether
the station
is the receiver of the packet (or one of the receivers, if the packet is
a broadcast one).

A message type is declared in the following way:
{\small
\begin{tabbing}
{\tt 12345678} \= {\tt 1234} \= {\tt 1234} \kill
\> {\tt message mtype : supmtype \{} \\
\> \> \ldots {\em non-standard attributes} \ldots \\
\> {\tt \};}
\end{tabbing} }
\noindent
where {\tt supmtype} is an already defined message type (it can be omitted
if the new type is derived directly from {\tt Message}).
An argument-less {\tt setup} method can be defined for a message type: it
will be called
whenever a new message is generated and queued at its sender station.
Note that messages are seldom generated directly by the protocol program.

Similarly, a packet type can be defined as:
{\small
\begin{tabbing}
{\tt 12345678} \= {\tt 1234} \= {\tt 1234} \kill
\> {\tt packet ptype : supptype \{} \\
\> \> \ldots {\em non-standard attributes} \ldots \\
\> {\tt \};}
\end{tabbing} }

The {\tt setup} function, if defined, is called whenever a packet of the
declared type is acquired from a message.
Its purpose is to set up the non-standard packet attributes.

A {\tt setup} function must be declared for a packet type, if this
type is non-trivially extended, i.e.,
it actually defines some non-standard attributes.
The function should expect one argument which should be a message
object pointer.
This argument points to the message from which the packet is being
acquired.

Typically, a station defines a number of packet buffers that are used to
store packets acquired for transmission, packets relayed to other
stations, etc.
A packet buffer is just an object of a packet type defined statically
within the scope of the station type definition.
A packet buffer can be either {\em full\/} (if it contains a packet) or
{\em empty\/} (if no packet is currently stored in the buffer).
Type {\tt Packet} defines a collection of
methods for determining the status of a packet buffer.

\subsubsection{Message arrival process}

The traffic in the modeled network is described as a set of independent
patterns, each pattern characterized by a specific distribution parameters
of the message arrival process.
\smurph\ offers standard tools for defining traffic patterns---objects
of type {\tt Traffic}.
Should these tools prove inadequate, the user can program his private
traffic patterns as extensions of type {\tt Traffic}.

A standard traffic pattern is described by the following parameters:
\begin{itemize}
\item
The distribution of senders and receivers.
This part determines how often particular stations are selected as senders
and receivers for messages generated according to the given pattern.
\item
Message inter-arrival time distribution.
These parameters determine how much time elapses between two consecutive
message arrival events.
\item
Message length distribution.
One attribute of a message is its length in bits.
For a given traffic pattern, this length can be either fixed or generated as
a random number according to the message length distribution.
\end{itemize}

The distribution of senders and receivers is given as the set of the so-called
{\em communication groups}.
A single communication group (an object of type {\tt CGroup}) defines two
sets of stations: the set of senders and the set of receivers.
Every station within each of the two sets is assigned a weight which can be
viewed as the relative probability that this station will be chosen.
The entire {\tt CGroup} has also a weight which determines the probability of
the group being used to generate the sender and the receiver.
The group weight is relevant when the traffic pattern is described by
more than one {\tt CGroup}.
With this two-level selection mechanism, it is possible to build refined
traffic patterns.
\smurph\ provides shortcuts for defining simple patterns: in many cases the
notion of a communication group need not be used.

A communication group is defined in terms of simpler objects called
{\em station groups} which belong to type {\tt SGroup}.
An {\tt SGroup} is simply a subset of stations.
In particular,
the set of receivers for a broadcast traffic pattern is specified as a
station group.

The procedure of generating a new message and queuing it at the sending
station consists of a number of steps, e.g.,
selection of a communication group, generation of the sender and the
receiver, generation of the message length.
Each of these steps is performed by a virtual method defined within type
{\tt Traffic}.
By extending this type, the user can substitute his own functions for
the standard methods.
Similarly, an extension of {\tt Traffic} may define private methods for
calculating various traffic pattern-related performance measures.

A traffic pattern type is declared in the following way:
{\small
\begin{tabbing}
{\tt 12345678} \= {\tt 1234} \= {\tt 1234} \kill
\> {\tt traffic ttype : supttype (mtype, ptype) \{} \\
\> \> \ldots {\em non-standard attributes and methods} \ldots \\
\> {\tt \};}
\end{tabbing} }
\noindent
where {\tt supttype} is an already defined pattern type (this part can be
omitted if the new traffic type is derived directly from {\tt Traffic}),
{\tt mtype} and {\tt ptype} stand
for the types of messages and packets to be generated according to
the pattern.
If these types are {\tt Message} {\tt Packet}, respectively, they (and
the parentheses enclosing them) can be skipped.
The base traffic pattern type {\tt Traffic} can be used directly for
traffic generation.
It generates messages of type {\tt Message} and packets of type
{\tt Packet}.

A traffic pattern type can be used to create specific traffic patterns.
The most general format of {\tt create} that serves this end is:
{\small
\begin{verbatim}
        create ttype (cgl, ncgl, flags, ...)
\end{verbatim} }
\noindent
where {\tt cgl} points to an array of pointers to communication groups,
{\tt ncgl} gives the number of communication groups pointed to by {\tt cgl},
and {\tt flags} is a collection of binary flags specifying distribution
modes for the message inter-arrival time and message length.
The remaining arguments (their number depends on the contents of {\tt flags})
give the numeric parameters of these distributions.

Very often a traffic pattern is described by a single communication group.
In such case, {\tt cgl} can be a direct pointer to the only
communication group and {\tt ncgl} should be omitted.
It is also possible to define a traffic pattern without specifying any
communication group and later build the sets of senders and
receivers dynamically.
This last method is recommended for simple traffic patterns.

\subsection{Processes}

The protocol behavior is described by a set of processes (objects of type
{\tt Process}) run at stations.
A process consists of its private data area, and code which can be shared
by a number of processes belonging to the same process type.
A process is always associated with some station: we say that the station
owns (or runs) the process.

Besides accessing its private data area, a process can reference the
attributes of the station owning the process and some global variables
constituting the so-called {\em process environment}.
Processes can communicate in several ways, even if they do not belong to
the same station.

A process type consists of a number of attributes (they can be viewed as
local variables of the process), an optional {\tt setup} method (which is
typically used to initialize local variables when a process instance is
created), and another method defining the process code.
The syntax of a process type declaration is:
{\small
\begin{tabbing}
{\tt 12345678} \= {\tt 1234} \= {\tt 1234} \= {\tt 1234} \kill
\> {\tt process ptype : supptype (fptype, stype) \{} \\
\> \> \ldots {\em attributes and methods} \ldots \\
\> \> {\tt setup (} ... {\tt ) \{} \\
\> \> \> \ldots \\
\> \> {\tt \};} \\
\> {\tt ~} \\
\> \> {\tt states \{s0, s1, } ... {\tt , sk\};} \\
\> {\tt ~} \\
\> \> {\tt perform \{} \\
\> \> \> \ldots \\
\> \> {\tt \};} \\
\> {\tt \};}
\end{tabbing} }
\noindent
where {\tt ptype} is the name of the declared process type,
{\tt supptype} is an already defined process type,
{\tt fptype} is the type of the process' {\em father}, and
{\tt stype} is the type of the station owning the process.
Similarly as for the other \smurph\ types, {\tt supptype} can be omitted
if the new process type is derived directly from {\tt Process}.
The two arguments in parentheses are also optional: they can be skipped
if they are irrelevant from the viewpoint of the new type.

A process is always created by another process which is considered its
father.
Initially, before any user-defined process is created, there exists a
system process called {\tt Kernel} (of type {\tt Process}) which is the
father of the first user process.

A process is created similarly as other dynamic objects, by the command:
{\small
\begin{tabbing}
{\tt 12345678} \= {\tt 1234} \= {\tt 1234} \kill
\> {\tt prcs = create ptype (} ... {\tt );}
\end{tabbing} }

As usual, the arguments enclosed in the second pair of parentheses are passed
to the process' {\tt setup} method.

A process can be terminated either by itself (by executing {\tt terminate~()})
or by another process (which executes {\tt prc->terminate~()}, where {\tt prc}
is the pointer to the process to be killed).

The keyword {\tt perform}\footnote{The resemblance to COBOL is
accidental.} begins the declaration of the process code method.
This method can be defined directly within the process type declaration or
it can be just announced there (if {\tt perform} is immediately followed
by a semicolon) and defined later in the following way:
{\small
\begin{tabbing}
{\tt 12345678} \= {\tt 1234} \= {\tt 1234} \kill
\> {\tt ptype::perform \{} \\
\> \> \ldots \\
\> {\tt \};}
\end{tabbing} }

The process code method resembles the description of a finite
state machine.
The {\tt states} declaration assigns symbolic names to the states of this
machine.
The first state on the list is the initial state: the process gets into this
state automatically after it is created.

The operation of a process consists in responding to various events triggered
by its environment.
The occurrence of an event awaited by a process wakes the process up and
forces it to a specific state.
Then the process (its code method) performs some operations and suspends
itself.
Typically, among these operations are indications of future
events that the process wants to perceive.
Thus, the process code method can be viewed as the transition function of
a finite state machine.
Most often, this method has the following structure:
{\small
\begin{tabbing}
{\tt 12345678} \= {\tt 1234} \= {\tt 1234} \= {\tt 1234} \= {\tt 1234} \kill
\> {\tt perform \{ } \\
\> \> {\tt state s0:} \\
\> \> \> \ldots \\
\> \> {\tt state s1:} \\
\> \> \> \ldots \\
\> \> \ldots \\
\> \> {\tt state sk:} \\
\> \> \> \ldots \\
\> {\tt \};}
\end{tabbing} }

Process code methods are not inherited from supertypes, i.e.,
each process type declares its code method from scratch.

Two standard pointers (whose declarations are invisible) are available to
a code method.
They are: {\tt F} (of type {\tt fptype}) pointing to the process' father
and {\tt S} (of type {\tt stype}) pointing to the station owning the
process.
This way, besides having natural access to its private objects, a process can
reference public attributes of its father and its station.

\subsection{Activity interpreters}

Operations of protocol processes are driven by events.
These events are generated by objects called {\em activity interpreters}.
Before suspending itself, a process declares events that will wake it
up in the future.
The occurrence of the earliest of these events restarts the
process.
If two or more of the awaited events occur at the same time
(within the same {\em ITU\/}),
one of them is selected at random.
Thus, a process is always awakened by exactly one event.

Activity interpreters model the time flow.
An {\em AI\/} is responsible for determining how much time a specific
physical phenomenon would take in the real system and advances the
clock of the system model by the corresponding number of {\em ITU\/}s.
The input to the {\em AI\/}s is provided by the protocol processes.
We say that the protocol processes exhibit activities which are absorbed
by {\em AI\/}s.
Based on these activities, the {\em AI\/}s predict future events and their
timing.

The interface between an {\em AI\/} and the protocol program depends on the
{\em AI}.
One element of this interface
is common for all {\em AI\/}s; it is the method:
{\small
\begin{verbatim}
        wait (ev, st)
\end{verbatim} }
\noindent
which a process can invoke to request to be awakened by a specific future event.
The first argument of {\tt wait} identifies the event; its type and range are
{\em AI\/}-specific.
The second argument is a state identifier: it says that upon the occurrence of
the indicated event the process wants to be awakened at the given state.

A restarted process usually performs a sequence of operations (possibly
exhibiting activities addressed to some {\em AI\/}s) and then it
issues a number of {\tt wait} requests describing its future waking
conditions.
Eventually, the process suspends itself---either by exhausting the list
of statements associated with its current state or by executing {\tt sleep}.
A process that puts itself to sleep without specifying at least one waking
condition becomes terminated.
All the {\tt wait} requests issued by a process at one state are combined into
an {\em alternative\/} of waking conditions.
It means that as soon as one of
these conditions is fulfilled, the process is restarted at the state previously
indicated by the corresponding {\tt wait} request.
The other conditions are then erased; thus,
if the process decides later to wait for the same (or a similar)
configuration of
events, all the {\tt wait} requests must be issued from the beginning.

While a process is running, i.e.,
from the moment the process was restarted until it suspends itself,
the simulated time does not flow.
This way multiple processes can be active {\em simultaneously\/}---within the
same {\em ITU}.
One of the basic principles of \smurph\ is that the modeled time only flows
when it is advanced by one of the {\em AI\/}s.

\subsubsection{The {\tt Timer} {\em AI}}

There is exactly one timer {\em AI\/} pointed to by the global variable
{\tt Timer}.
By issuing a
{\tt wait} request to the timer, e.g.:
{\small
\begin{verbatim}
        Timer->wait (interval, ready);
\end{verbatim} }
\noindent
a process declares that it
wants to be restarted at state {\tt ready}, {\tt interval}
{\em ITU\/}s after the current moment.
Another part of the timer interface is the global variable {\tt Time} which
tells the current virtual time as the number of {\em ITU\/}s elapsed since
the beginning of simulation.

The {\tt Timer} wait request is implicitly used to implement two operations
for unconditional branching to another process state.
Namely, the following two operations:
{\small
\begin{tabbing}
{\tt 12345678} \= {\tt 1234} \= {\tt 1234} \= {\tt 1234} \= {\tt 1234} \kill
\> {\tt proceed stt;} $\;\;\;\;\;\;$ and $\;\;\;\;\;$ {\tt skipto stt;}
\end{tabbing} }
\noindent
are equivalent to:
{\small
\begin{verbatim}
        Timer->wait (0, stt);
        sleep;
\end{verbatim} }
\noindent
and
{\small
\begin{verbatim}
        Timer->wait (1, stt);
        sleep;
\end{verbatim} }
\noindent
respectively.
The first operation is used to branch unconditionally to the indicated state
within the same {\em ITU}.
The second operation delays the branching by one {\em ITU} and is useful
for skipping certain events that remain pending until the time is advanced.

It is possible to request that all time intervals be determined with a
certain station-dependent tolerance---to simulate the limited accuracy of
real independent clocks.

\subsubsection{{\tt Traffic} {\em AI\/}s and the {\tt Client} {\em AI}}

Each object belonging to type {\tt Traffic} can be viewed as an independent
{\em AI\/} providing stations with messages to transmit.
Moreover, there is one {\tt Client}
{\em AI} which can be viewed as a combination
of all individual {\tt Traffic} {\em AI\/}s.
A process may poll these {\em AI\/}s for a packet to be transmitted.
If no packet is available, the process may decide to suspend itself until
a message arrives at the station.
Let us assume that {\tt tp} points to an object of type {\tt Traffic}.
The following operation:
{\small
\begin{verbatim}
        gotit = tp->getPacket (buf, min, max, frm);
\end{verbatim} }
\noindent
attempts to acquire a packet from the first-arrived
message belonging to the traffic pattern {\tt tp}.
If the attempt is successful, the function returns {\tt YES}; otherwise,
if no message of pattern {\tt tp} is queued at the station, {\tt NO} is
returned.
In case of success, the packet is put into the buffer pointed to by
{\tt buf}.
The length of the packet's information part is never less than {\tt min}
(dummy bits are added if the message is too short to fill the
packet) nor is it greater than {\tt max} (only a part of the message is used
if the message is longer that {\tt max}; the remaining part remains queued).
The last argument of {\tt getPacket} gives the length of the frame part, i.e.,
the number of additional bits to be added to the packet to furnish it with
the pertinent header and trailer.

In the case when the traffic pattern of the packet to be acquired is
irrelevant, the {\tt Client} {\em AI\/} can be used.
For example, by calling:
{\small
\begin{verbatim}
        gotit = Client->getPacket (buf, min, max, frm);
\end{verbatim} }
\noindent
a process polls the {\tt Client}
for a packet belonging to any traffic pattern.
In this case, the first-arrived of all messages queued at the station
is used, irrespective of its pattern.
Thus, this call only fails, if all message queues at the station are empty.

Should an attempt to acquire a packet fail, a {\tt wait} request addressed
either to a specific object of type {\tt Traffic} or to the {\tt Client}
can be issued to awake the process when a message is generated and queued
at the station.
In particular, with the following request:
{\small
\begin{verbatim}
        Client->wait (ARRIVAL, tryagain);
\end{verbatim} }
\noindent
a process declares that it
wants to be awakened at state {\tt tryagain} as soon as a message
(of any pattern) arrives at the station.

\subsubsection{{\tt Port} and {\tt Link} {\em AI\/}s}

Functionally, {\tt Link} objects are distributed and
represented by collections of ports, each port being visible to the
protocol program as a separate {\em AI}.
Internally,
each link can be viewed as a single {\em AI\/} with centralized
processing.
Protocol processes seldom reference links directly, with exception of the
topology initialization phase.
A protocol process may access a port {\em AI\/} for one of the following
three reasons:
\begin{itemize}
\item
to change the port status, e.g.,
to start or terminate a packet transmission or a jamming signal;
\item
to inquire the port about its present or past status;
\item
to issue a {\tt wait} request to the port, i.e.,
to await a moment when the port gets into a specific state.
\end{itemize}

There are two types of activities that can be inserted into ports:
{\em packet transmissions\/} and {\em jamming signals}.
The latter are used in protocols based on {\em collision detection\/} to
enforce the so-called {\em collision consensus} (cf.\ 
\cite{meb76}).
Jamming signals are in some sense redundant: they can be simulated by special
packets; however, their presence makes it easier to program an important class
of MAC-level protocols.

An activity, be it a packet transmission or a jamming signal, must be explicitly
started and explicitly terminated.
Assuming that {\tt pt} represents an object of type {\tt Port}, the following
operation:
{\small
\begin{verbatim}
        pt->transmit (buf, done);
\end{verbatim} }
\noindent
initializes a packet transmission on port {\tt pt}.
The packet is taken from the buffer pointed to by {\tt buf}.
After the packet has been completely transmitted, the process will be
awakened at state {\tt done}.
The fact that the packet has been completely transmitted does not imply that
its transmission has stopped.
The process is restarted at state {\tt done} after the amount of time
equal to the total length of the packet
multiplied by the port transmission rate.
Then, the process is supposed to stop the transmission by calling:
{\small
\begin{verbatim}
        pt->stop ();
\end{verbatim} }
\noindent
usually followed by:
{\small
\begin{verbatim}
        buf->release ();
\end{verbatim} }
\noindent
which empties the packet buffer and updates certain performance measures.

Typical port inquiries are functions returning the time when the port was
last found in some specific state.
For example:
{\small
\begin{verbatim}
        t0 = pt->lastBOT ();
\end{verbatim} }
\noindent
assigns to {\tt t0} the time when the last beginning of packet was heard on
port {\tt pt}.

A port {\tt wait} request may identify one of the following future events:
\begin{llist}{{\tt COLLISIONN}}
\litem{{\tt SILENCE}}
the earliest beginning of a silence period (the event occurs in the current
{\em ITU} if the port is already silent);
\litem{{\tt ACTIVITY}}
the earliest beginning of an activity period;
\litem{{\tt COLLISION}}
interference of two or more packet transmissions or the beginning of a jamming
signal;
\litem{{\tt BOT}}
the beginning of a packet;
\litem{{\tt EOT}}
the end of a packet;
\litem{{\tt BMP}}
the beginning of {\em my\/} packet, i.e.,
a packet addressed to the station whose process issues the {\tt wait} request;
\litem{{\tt EMP}}
the end of {\em my\/} packet;
\end{llist}
and a few others.

If a process is awakened by an event that has been triggered by a packet
transmission, the variable {\tt ThePacket} (of type {\tt Packet}) points to
packet object and another variable, {\tt ThePort}, points to the
port on which the packet is heard.
These two global variables (as well as a few other variables settable by
various events) belong to the {\em process environment}.
Their purpose is to pass to the awakened process some information related to
the restarting event.

\subsubsection{{\tt Mailbox} {\em AI\/}s}

{\tt Mailbox} {\em AI\/}s provide means for inter-process communication.
A mailbox, which is always owned by some specific station, can be viewed as
a depository for messages passed among the processes running at the station.
The capacity of a mailbox is limited and
determined at its creation.
Depending how a mailbox is defined and created
(the standard type {\tt Mailbox} can be extended by the user), its functionality
may resemble a simple signal (interrupt) passing mechanism or a FIFO-type
storage for compound objects.
The standard type {\tt Mailbox} (without extension)
can be used to pass simple signals with no information content.

A mailbox can be declared statically, as part of a station type definition,
or created dynamically within the context of the station to which it is
supposed to belong.
One {\tt setup} argument that can be specified upon mailbox creation is
the capacity.
If no capacity is specified when the mailbox is created, default capacity
$0$ is assumed.
This means that no items will ever be
stored in the mailbox: if no process is waiting for an item and one arrives,
it is ignored.

Besides the {\tt wait} method, the {\tt Mailbox} {\em AI\/} offers two
operations: {\tt put}---to send a new item to the mailbox and {\tt get}---to
retrieve the front element from the mailbox queue.

By issuing the following mailbox {\tt wait} request:
\begin{verbatim}
        mb->wait (NEWITEM, gotit)
\end{verbatim}
a process awaits the moment when a new item is put into the mailbox.
By using {\tt RECEIVE} instead of {\tt NEWITEM}, the process can request to
automatically accept the item when it arrives.
In such case, the value of the item will be returned to the process via
the environment variable {\tt TheItem}.

Another example of an event that can be specified as the first argument of
the {\tt wait} method is a nonnegative number $n$.
The event is triggered when the number of elements stored in the mailbox reaches
$n$.

\subsubsection{{\tt Process} {\em AI\/}s}

One more way of communicating processes is to take advantage of the fact
that each protocol process is an independent {\em AI\/}.
Assume that {\tt prc} is a pointer to a process object.
By issuing the following request:
{\small
\begin{verbatim}
        prc->wait (prcstate, mystate);
\end{verbatim} }
\noindent
a process requests to be awakened at {\tt mystate} as soon as the
process pointed to by {\tt prc} reaches {\tt prcstate}.
One special event that can be specified as the first argument of the
{\tt Process} {\tt wait} method is {\tt DEATH} which represent the termination
of the process in question.
For example, by executing the following sequence:
{\small
\begin{tabbing}
{\tt 12345678} \= {\tt 1234} \= {\tt 1234} \= {\tt 1234} \= {\tt 1234} \kill
\> {\tt prc = create mychild (} ... {\tt );} \\
\> {\tt prc->wait (DEATH, done); } \\
\> {\tt sleep; }
\end{tabbing} }
\noindent
a process creates a child process and blocks itself until the child is
terminated.
Thus, the child process can be viewed as a {\em subroutine\/} called by its
creator, as opposed to the situation when the child and its father operate
concurrently.

\subsection{User interface}

Typically, the protocol program reads some input data, although one can
imagine a situation in which no input data is required, i.e.,
all parameters are hard-coded into the program.
\smurph\ offers private operations for reading numbers from input
(the standard C++ tools can be used as well) which ignore all non-numeric
contents of the input file and provide shortcuts for reading sequences of
related numbers.
It is easy to insert textual comments into the data file: they are
just bypassed and ignored.

A similar collection of basic output tools is provided.
In most cases, however, the output results are produced by exposing, i.e.,
calling high-level
output methods associated with particular objects.
For example, assuming that {\tt tpt} points to a traffic pattern,
the call:
{\small
\begin{verbatim}
        tpt->printPfm ();
\end{verbatim} }
\noindent
writes to the output file the set of performance measures associated with
the pattern {\tt tpt}.

Usually, a single object defines a number of {\em exposing\/} methods;
some of them can be used for debugging.
In most cases, these methods can be subsumed or augmented by user-defined
methods in type extensions.

Another part of the \smurph\ user interface is the
{\em dynamic status display\/}
which allows the user to monitor the protocol behavior on-line.
The display is handled by a separate, in principle exchangeable, program
which communicates
with \smurph\ using a well-defined and reasonably simple protocol.
In particular, it is possible to run the simulator on one machine, e.g.\ 
a CPU server, and monitor its execution on another computer, e.g.\ 
a graphic workstation.

\subsection{Performance measures}

The global performance of a network is usually expressed as the correlation
of {\em throughput} and {\em delay}.
By default,
the performance statistics collected by \smurph\ during simulation
include three different measures of delay for each traffic pattern; they are:
the packet delay (excluding the message queuing time),
the absolute message delay which includes the message queuing time, and
the weighted message delay which includes the message queuing
time and reflects the fact that long messages may be split into multiple
packets reaching their destination at different times.
Each measure is presented in the form of a random variable (an object of
type {\tt RVariable}) with a number of
parameters including the minimum, maximum, mean, and the standard
deviation.
The throughput is measured separately for each link and globally for the
entire network.

\smurph\ offers tools for collecting non-standard statistics.
Objects of type {\tt RVariable} can be created by the user and processed by
a collection of standard methods implementing typical operations
on random variables, e.g., adding a new sample,
combining two random variables into one, and printing out or displaying the
parameters of a random variable.

\subsection{Observers}

Besides conventional tools for program debugging, as local
assertions and protocol tracing functions, \smurph\ offers means for
verifying compound dynamic conditions that can be viewed as an alternative,
static specification of the protocol.

Observers are process-like objects that can be used for expressing global
assertions that involve the combined behavior of more than one regular process.
An observer may specify that it is to be awakened whenever a
process is restarted at a specific state.
A similar approach to implementing self-checking distributed programs
was proposed earlier (\cite{aad79}) and recently refined (\cite{gro86}).
Observers can be seen as a tool for writing non-executable protocol
specifications in terms of dynamic formulas describing possible successions
of protocol states.

\documentstyle [amssymbols,11pt] {article}
\input {../../../MANUAL/rm_en}

\begin{document}

\begin{titlepage}

\vspace*{3.5in}
\begin{center}
\Huge\bf ETHERNET \vspace{0.25in}
\end{center}
\end{titlepage}

\section{The protocol}

The following Ethernet protocol is based on the commercial
network according to (IEEE 802 Local Network Standard):
\begin{llist}{EEE}
\litem{E1}
The network consists of a number of stations distributed along the bus.
The bus is a single link and forms a tree with the stations being the leaves.
Each station has a single {\em port} connecting it to the bus.
\litem{E2}
All stations constantly monitor the bus and are able to perceive its
status.
At any given moment there may be
several activities on the bus, each started by a different station.
If a station can sense any of these activities
then it assumes that the bus is {\em busy}.
\litem{E3}
A station acquiring a packet to transmit senses the bus.
If the bus has been {\em idle} for the amount of time equal to the
{\em inter-packet space}, the station starts transmitting the packet.
If the bus is sensed idle, but the silence period has been too short,
the station waits until the inter-packet space is obeyed and then starts to
transmit.
If the bus is busy, the station waits until it becomes idle, then obeys the
inter-packet space and transmits.
\litem{E4}
A transmitting station keeps sensing for a {\em collision\/}---an
interference caused by some other activity on the bus.
If a collision has been detected, the station
aborts the transfer immediately and sends a
{\em jamming signal} to make sure that
any other party involved in the collision also recognizes it.
Then the station reschedules the retransmission attempt of its packet after
a randomized time delay determined by the {\em backoff algorithm}.
\litem{E5}
There is an upper limit on the length of a packet which can be sent in one
transfer.
Any message longer than the maximum packet size is split into several packets.
Each packet is
augmented by fixed-length {\em frame information} (header and trailer).
There also is a lower limit on the
packet length---to make sure that no packet is shorter than
$2t_a$ (the round-trip propagation delay of the bus expressed in bits).
This way all collisions are detectable during the
transmission time, and
as soon as the end of packet is transmitted, the station may stop listening for
a collision.
Packets shorter than that limit are expanded to the minimum legal size.
\end{llist}

Ethernet is a uniform network, in the sense that all stations obey the
same protocol and transmit with the same speed.
Thus, it is natural and convenient to assume that all distances and time
intervals are expressed in bits.
For example, in the commercial network $1$ bit $= 10^{-7}$
second $\approx $15m.

The numerical values of the relevant parameters for the protocol outlined above
are given in the following table.
Note that all values are expressed in bits.

\medskip

\begin{center}
\begin{tabular}{|lr@{\ }l|}
\hline 
Data rate		& 10	&Mbps\\
Minimum packet size	& 368	&bits\\
Maximum packet size	& 12000 &bits\\
Frame information	& 208 	&bits\\
Minimum packet spacing	& 96 	&bits\\
Minimum jam size	& 32 	&bits\\
Maximum bus length	& 256 	&bits\\ \hline
\end{tabular}
\end{center}

\medskip

The algorithm used to reschedule the retransmission after a colliding transfer
attempt is the so-called {\em Binary Exponential Backoff\/} ({\em BEB\/}). The
delay is generated as:
$2t_a \times U(0,2^{n_c} )$,
where $t_a$ is the maximum end-to-end propagation delay of the bus ($256$ bits),
$U$ denotes a uniformly distributed random number from the specified
interval, and $n_c$ is the number of collisions (unsuccessful transfer
attempts) suffered by the current packet.
It is assumed than if $n_c$ exceeds $10$, an exception condition is passed to
the higher protocol layers.

\section{The protocol code}

\subsection{Data structures and network topology}

An Ethernet station is described by type {\tt Node} having two non-standard
attributes: a single packet buffer and a port connecting the station to the
bus.
The bus is represented by a single, bidirectional, broadcast link
(an object of type {\tt Link}) created by the {\tt Root}
process.
Note that the archival time for the bus link is non-zero.
The transmitter process (see below) takes advantage of port inquiries to
the bus to determine whether the packet space has been obeyed.
The archive time is equal to the time duration of the packet space augmented
by a ``safety margin".

It is assumed that the bus geometry is strictly linear: the distance between
two neighboring stations (or rather their ports) is the same and equal to
{\tt (BusLength~*~TGran)/(NNodes-1)}, where {\tt BusLength} is the (one-way)
length of the bus expressed in bits, {\tt TGran} is the {\em time granularity},
i.e., the number of {\em ITU\/}s corresponding to the time represented by a
single bit-slot.

Note that although the bus is strictly linear, it is represented by an
object of type {\tt Link} (not {\tt PLink}), as it is a bidirectional,
broadcast channel.
Thus, the distance matrix for the bus must be specified completely
(it is done at state {\tt Start} in the {\tt Root} process), i.e.,
the distance between each pair of ports has to be explicitly given.

Each station runs two processes: the transmitter ({\tt Xmitter}) that takes
care of acquiring packets for transmission, recognizing collisions,
backing off, etc., and a very simple receiver ({\tt Receiver}) that just
accepts packets addressed to its station and executes {\tt receive} for
them.

\subsection{The transmitter}

Upon setup, the transmitter process creates its private copy of the
port pointer (denoted by {\tt Bus}) and the pointer to the
station's packet buffer.
It also converts some time parameters specified in bits to {\em ITU\/}s,
so that they can be used more efficiently, without dynamic conversion in
the process' code.

The operation cycle of the transmitter starts at state {\tt NPacket} where
the process attempts to acquire a packet for transmission.
If this attempt fails, the process waits for a {\tt Client}'s message
{\tt ARRIVAL} event; otherwise, it clears the collision counter
(attribute {\tt CCounter}), which will be used to count the number of
collisions suffered by the newly acquired packet, and continues at
state {\tt Retry}.

The first operation performed at {\tt Retry} is determining the time
when the {\tt Bus} was last sensed idle.
The method {\tt lastEOA} returns the time when the last activity on the
port was terminated, or an undefined value (recognized by the predicate
{\tt undef}) if the port is currently perceiving some activity, i.e., the
{\tt Bus} is busy.
In such case, the transmitter waits for a {\tt SILENCE} on the port and tries
again.

If the bus is idle, the value of {\tt IPeriod} calculated as the difference
between the current time ({\tt Time}) and the time when the current silence
period began ({\tt LSTime}) tells the length of the silence period.
This length must not be shorter than the required packet space interval
({\tt PSpace}) for the transmission to commence.
Note that {\tt lastEOA} monitors the past activities in the bus only down
to the period determined by the bus' archival time.
It is sufficient for this period to be only as long as {\tt PSPace}:
anything that happened in the bus earlier than {\tt PSPace} time units
before the current moment is irrelevant and can be treated as silence.

If the idle period turns out to be too short, the transmitter waits for
the amount of time corresponding to the difference between the required
packet space and the actual length of the silence period.
Otherwise, it moves directly
to state {\tt Xmit} where the packet transmission is started.

Having started the packet transmission, the process issues a wait request
to the {\tt Bus} port to be notified about a possible collision.
If the packet has been transmitted successfully without a collision,
the transmitter gets to state {\tt XDone} where it terminates the
transfer, releases the packet (empties the packet buffer), and moves
back to state {\tt NPacket}---to acquire another packet for transmission.
If a collision occurs, the process wakes up at state {\tt XAbort} where
it aborts the transfer, increments the collision counter, and starts
sending a jamming signal.

After the amount of time needed to send the jamming signal, the transmitter
is restarted at state {\tt JDone}.
There it stops jamming, waits for the amount of time determined by the
backoff function, and moves to {\tt Retry}---to retransmit the packet.
Note that the backoff function directly implements the formula from the
first section.
The collision counter used by the backoff function is forced not to exceed
10.

\subsection{The receiver}

The receiver process is very simple and it needs little explanation.
The only {\tt Bus} event interesting from the viewpoint of the receiver
is {\tt EMP} ({\em End of My Packet\/}) representing a complete and
successful reception of a packet addressed to the station.
Note that packets aborted (due to collisions) do not generate {\tt EMP}
events.

Whenever the process is awakened by an {\em EMP} event, it executes
{\tt receive} for the packet and moves back to state
{\tt WPacket} to await another such event.
Note that the receiver uses {\tt skipto} (as opposed to {\tt proceed}),
to make sure that the {\tt EMP} event has disappeared from the port.
Otherwise, the process would loop infinitely on the same {\tt EMP}
event without advancing the modeled time.

\end{document}

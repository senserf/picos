\documentstyle [amssymbols,11pt] {article}
\input {../../../MANUAL/rm_en}

\begin{document}

\begin{titlepage}

\vspace*{3.5in}
\begin{center}
\Huge\bf A Disk Model \vspace{0.25in}
\end{center}
\end{titlepage}

\section{The model}

This program models a simple disk consisting of two physical items:
the disk drive and the disk controller.
We assume that the disk controller receives requests from the outside world
(modeled by the \smurph\ {\tt Client})
and passes them to the disk drive for processing.
Two request types are considered: read requests and write requests.
With a write request, the controller sends to the disk drive the new contents
of a sector to be written.
A read request identifies a sector whose contents are to be read from the
disk and returned to the controller.

The amount of time spent by the disk drive on processing a request depends
on the current position of the heads (the seek time) and on the location
of the addressed sector within the track (the latency).
The disk drive keeps track of the current heads position and simulates the
rotation of the disk platters.

\section{The implementation}

\subsection{The topology}

The network implementing the disk model consists of two stations:
the controller and the disk drive which are connected by a link.
Each of the two stations has one port to the link.
The controller uses its port to pass requests to the disk drive,
the disk drive sends on its port responses to the controller.

\subsection{The traffic}

A special message type {\tt Request} is introduced to represent disk
requests arriving from the {\tt Client}.
This message type has two non-standard attributes:
{\tt SecAddr} specifying the sector number and {\tt Type} identifying the
request type ({\tt READ} or {\tt WRITE}).

The controller turns such a message into a packet of type {\tt Preamble}
which describes the request for the disk drive.
This packet is transmitted to the disk drive on the single link
connecting the two components of our system.
If the request is a {\tt WRITE} operation, the {\tt Preamble}
is followed by a packet of type {\tt Sector} representing the data to
be written.
The only relevant attribute of this packet is its length (equal to the
sector length) which determines the amount of time needed to transmit and
write the sector contents.
A {\tt READ} request is completely characterized by the {\tt Preamble}.

An operation completed by the disk drive results
in a reply being sent to the
controller.
With the completion of a {\tt READ} operation, the drive responds with
the sector contents transmitted as a packet of
type {\tt Sector}.
This packet is immediately followed by another short packet of type
{\tt Response} that represents the status information returned by the
disk drive to the controller.
The reception of this status packet marks the completion of the i/o
operation on the controller's end.
The {\tt Response} packet is the only reply sent to the controller
after processing a {\tt WRITE} request.

The behavior of the standard {\tt Client} is driven by two traffic patterns:
one handling the {\tt READ}-type requests and the other generating
the {\tt WRITE}-type requests.
All messages representing requests are queued at the controller and are
addressed by the {\tt Client} to the disk drive.

The standard {\tt Client} is responsible for determining inter-arrival intervals
between requests and setting the message and packet length attributes.
The message length for each of the two patterns is the same: it represents
the length of the request {\tt Preamble} that each message is turned into.
Two more message attributes must be determined:
the sector number and the request type.
This is done by the {\tt RqstGen} process which intercepts client messages
and augments them with a randomly generated (function {\tt toss}) sector
number.
The request type is directly inferred from the traffic pattern generating
the message.
Actually, the request type attribute is redundant (the traffic pattern
{\tt Id} is inherited by the packet sent to the disk); it exists merely for
clarity.

\subsection{The protocol}

Preprocessed messages are retrieved by the {\tt Handler} process
(running at the controller) which
turns them into requests (one at a time) and transmits to the disk drive.
A {\tt WRITE} request is followed by a {\tt Sector} packet representing the
image of the new sector.
Having sent a complete request to the disk drive, the {\tt Handler}
awaits the response which comes as one packet (for {\tt WRITE})
or two packets (for {\tt READ}).
When the complete response arrives, the process updates the random variable
{\tt Delay} by including a new service time sample.
The service time is calculated as the difference between the time when
the response from the disk drive is received and the time when the
request was generated and queued at the controller.
This interval is expressed in {\em ETU\/}s (intentionally representing
real seconds).

The disk drive (a station of type {\tt Disk})
runs two permanent processes {\tt Drive} and {\tt Service},
the latter occasionally spawning two more processes ({\tt Seek} and
{\tt Find}).
The objective of {\tt Drive} is to model the rotation of disk platters on the
spindle.
The process wakes up every time interval needed to advance the heads to a new
sector and inserts a dummy item into the mailbox {\tt NewSector}.
Note that the mailbox is created with the queue size of $0$.
Thus the operation of storing an item into the mailbox can be viewed as
a generation of a simple interrupt signal indicating a new sector
boundary.
This signal is perceived by the {\tt Find} process (see below) and used to
time the operation of accessing a specific sector on a given track.
With each signal, the {\tt Drive} process
updates the {\tt Disk}'s attribute {\tt CurSec} telling the current
track-relative sector number.

Whenever the {\tt Service} process gets a new request from the {\tt Handler},
it calculates the sector parameters and initiates the seek operation by
starting a special process ({\tt Seek}) to do the job.
The {\tt Seek} process determines the amount of time needed to perform the
seek operation and triggers an interrupt after that time.
Again, the interrupt comes as a dummy item being stored in a mailbox
pointed to by {\tt Positioned}.
Having completed the modeled seek operation the process terminates.
The seek time is calculated by function {\tt seekTime} associated with the
{\tt Disk}.
For simplicity, we assume that this time consists of three components:
the fixed startup time, the moving time proportional to the number of
cylinders swept, and the fixed braking time.

If the request is {\tt WRITE}, the {\tt Service} process reads the sector
image arriving from the controller.
This is done in parallel with the seek operation.
Otherwise, for a {\tt READ} request, nothing more is expected from the
controller and the process just waits until the seek operation
is completed.

The next stage is to wait until the sector to be accessed gets under the
heads.
Again this is done by a special process called {\tt Find} which
is created by {\tt Service} dynamically.
The process monitors the sectors passing by (the wait request to the
{\tt NewSector} mailbox) and it
terminates itself when the sector in question arrives.
Then the {\tt Service} process simulates the operation to be performed.
If the operation is {\tt WRITE}, the process waits until the sector is
gone.
For a {\tt READ} operation, a {\tt Sector} packet is transmitted to the
controller.
Note that the port transmission rate ({\tt ChannelBandwidth}) is defined as
the ratio of the sector advancing time to the sector length.
This way, the transmission of a {\tt Sector} packet takes the same
amount of time as reading or writing a disk sector.
Finally, the {\tt Service} process sends a {\tt Response} packet indicating
the completion of the transfer.
\end{document}

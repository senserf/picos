\section{An implementation of Hubnet}

In this section, we present a complete example of a network and its
protocol programmed in \smurph.
This network is {\em Hubnet\/} (cf.\ 
\cite{htw86}) which is a star-shaped architecture based on broadcast-type
communication.
The central station in this star is a switching device responsible for
resolving contention to the broadcast channel.

\subsection{The protocol}

{\em Hubnet\/} consists of a number
of regular stations connected to a central
switching device---the so-called {\em Hub}.
Each regular station is connected to the {\em Hub} via
two channels: the broadcast channel and the selection channel.
All the broadcast channels are connected
together and 
form a uniform broadcast medium (a single link).
The selection channels are
not connected: each selection channel is a separate link attaching one
regular station to the {\em Hub}.

The broadcast channel can be in one of two states which are recognized and
toggled by the {\em Hub}.
These states are: {\em busy\/}, meaning that a packet is being transmitted
on the broadcast medium, and {\em idle\/}, when, according to the {\em Hub}'s
perception, the broadcast channel is inactive.

A station willing to
transmit a packet sends it on the selection channel.
When the packet arrives at the {\em Hub}, the {\em Hub} determines the state of the
broadcast link.
If the broadcast link
is {\em idle}, the {\em Hub} connects the selection channel to the broadcast link and
marks the broadcast link as {\em busy}.
Thus, the packet is relayed on the broadcast channel and it will be heard by all
stations in due time.
As soon as the transmission is complete, i.e.,
the {\em Hub} detects silence on the selection channel,
the channel is disconnected from the broadcast link and
the broadcast link status is changed to {\em idle}.

If a packet arrives on a selection link while the broadcast channel is
{\em busy}, the packet is simply ignored.
The transmitting
station listens for the echo of its packet on the broadcast channel.
If the echo does not arrive after some time (depending on the distance
of the station from the {\em Hub}), the station assumes that its transmission
has been blocked and tries again.

\subsection{Startup and termination}

\smurph\ expects that the user-supplied protocol description defines a special
process type named {\tt Root}.
The simulator will create exactly one process of this type.
This process is responsible for building the network model,
creating the protocol processes, and detecting the end of simulation.
This part of our implementation of {\em Hubnet\/} is listed below.
{\small
\begin{verbatim}
process Root {

  Traffic *TPat;
  Hub     *TheHub;

  void  initTopology (), initTraffic (), initProtocol ();

  states {Start, Exit};

  perform {
     state Start:
       setEtu (1000);
       setLimit (200, 1000000000);
       initTopology ();
       initTraffic ();
       initProtocol ();
       Kernel->wait (DEATH, Exit);
     state Exit:
       TPat -> printPfm ();
  };
};
\end{verbatim} }

The {\tt Root} process has two states.
The first state ({\tt Start}) is triggered automatically when the process is
created.
At this state, the process sets up the relation between the {\em ETU\/} and
{\em ITU\/} (in this case one {\em ETU\/} is declared to be equal to $1000$
{\em ITU\/}s), defines the simulation exit condition (the simulation will stop
as soon as $200$ messages are completely received at their destinations or
the modeled time reaches $1,000,000,000$ {\em ITU}s, whichever happens first),
and calls three user-defined functions.
The first of these functions builds the network, the second defines
the traffic conditions, and the third starts the protocol execution.
Then, the process awaits the {\tt DEATH} of an internal process called
{\tt Kernel}.
This event marks the end of the simulation run and when it happens
the {\tt Root} process is restarted at state {\tt Exit}.

Variable {\tt TPat} is set
by the function {\tt initTraffic} (see below) to point
to the only pattern describing the traffic conditions in the network.
By invoking the {\tt printPfm} method of this object, {\tt Root} prints out
the standard set of performance measures associated with the traffic pattern.
Then the simulation run is actually finished.

\subsection{Station types}

The network consists of two different types of stations: regular stations
(which are assumed to be homogeneous), and the {\em Hub} acting as a
switching device.
The following initial part of the protocol program file
defines (among other things) the station data types:

{\small
\begin{verbatim}
identify (Hubnet);

int     NUsers;
TIME    EchoTimeout,
        HdrRecTime;
long    MinPL, MaxPL, FrameL, TRate;

enum {Idle, Busy};

station Hub {

  int    Status;
  Port   **HPorts;

  void setup () {
      HPorts = new Port* [NUsers+1];
      for (int i = 0; i <= NUsers; i++)
        HPorts [i] = create Port (TRate);
      Status = Idle;
  };
};

station User {

  Port    *SPort, *BPort;
  Mailbox *StartEW, *ACK, *NACK, *Timeout;
  Packet  Buffer;

  void setup () {
      StartEW = create Mailbox (1);
      ACK = create Mailbox (1);
      NACK = create Mailbox (1);
      Timeout = create Mailbox (1);
      SPort = create Port (TRate);
      BPort = create Port (TRate);
  };
};
\end{verbatim} }

The operation {\tt identify} is used to assign a name to the protocol.
This name will be included in the header of the output file.

Variable {\tt NUsers} (its contents are read from the input file by
{\tt initTopology})
contains the number of regular stations in the network.
A regular station is represented by an object of type {\tt User}.
Type {\tt Hub} is used to represent the central switching station.

Each regular station has two ports: one to the selection link (pointed to
by {\tt SPort}), the other ({\tt BPort}) to the broadcast link.
One packet buffer ({\tt Packet}) is used to store a packet awaiting
transmission.

The four mailboxes created at each regular station will be used to synchronize
the station's processes.
All the mailboxes are created with capacity $1$.
They represent simple signal passing devices, each device capable of storing
one pending signal.

The {\em Hub\/} has {\tt NUsers}$+1$ ports pointed to by the entries in
array {\tt HPorts}.
Port number $0$ connects the {\em Hub\/} to the broadcast channel, the
remaining ports represent the {\em Hub\/}'s perception of the selection
links connecting it to the regular stations.
One additional attribute of the {\em Hub\/} is the integer variable
{\tt Status} assuming two enumeration values {\tt Idle} and {\tt Busy}.

Ports (and also mailboxes) are created upon {\tt setup} of their stations.
Variable {\tt TRate} (initialized from the input data file) is the
transmission rate of the network (common for all ports).
The role of the remaining global variables declared within the above
code fragment is explained below.

{\tt EchoTimeout} represents the amount of time that elapses until the
transmitter of a packet assumes that the packet has not made it through the
{\em Hub}, if the packet echo does not appear in the meantime on the broadcast
port.
{\tt HdrRecTime} denotes the time needed to recognize the packet's sender.
A station awaiting the echo of its packet must wait for {\tt HdrRecTime}
since the moment it detects the beginning of a packet, until it can definitely
tell whether the packet was transmitted by the station.
{\tt MinPL}, {\tt MaxPL}, and {\tt FrameL} determine the packet framing, i.e.,
the minimum packet length, the maximum packet length, and the combined
length of the packet header and trailer, respectively.

\subsection{Network configuration}

The network configuration is defined by the following function:
{\small
\begin{verbatim}
void    Root::initTopology () {

  long    LinkLength;
  Link    *slk, *blk;
  User    *s1, *s2;

  readIn (NUsers);
  readIn (LinkLength);
  readIn (TRate);

  blk = create Link (NUsers+1);
  TheHub = create Hub;
  TheHub->HPorts [0]->connect (blk);

  for (int i = 1; i <= NUsers; i++) {
    slk = create Link (2);
    s1 = create User;
    s1->BPort->connect (blk);
    s1->SPort->connect (slk);
    TheHub->HPorts [i]->connect (slk);
  }

  for (i = 1; i <= NUsers; i++) {
    s1 = (User*) idToStation (i);
    s1->BPort->setDTo (TheHub->HPorts [0], LinkLength);
    s1->SPort->setDTo (TheHub->HPorts [i], LinkLength);
    for (int j = i+1; j <= NUsers; j++) {
      s2 = (User*) idToStation (j);
      s1->BPort->setDTo (s2->BPort, LinkLength+LinkLength);
    }
  }
};
\end{verbatim} }

The function
starts with reading in the number of regular stations ({\tt NUsers}),
the length of a link segment connecting a regular station with the {\em Hub}
(we assume that all these segments are of the same length), and the
transmission rate (common for all ports).

Then {\tt initTopology} creates the broadcast link (with {\tt NUsers}$+1$ ports)
and the {\em Hub} station.
The broadcast port of the {\em Hub\/} is connected to the broadcast link.

The first {\tt for} loop creates the {\tt NUser} selection links and
regular stations.
Each selection link has two ports.
This loop also connects the two station ports to the appropriate links
and the other end of the selection link to the {\em Hub}.

The next two loops assign distances between all pairs of ports of each
link.
The outer loop executes for all regular stations: it defines the distance
between a regular station and the {\em Hub}.
The inner loop defines the distance between a pair of selection ports
of different regular stations.
In the symmetric
star topology, this distance is equal twice the distance of a regular
station from the {\em Hub}.

\subsection{Traffic definition}

The traffic in our network is described by one traffic pattern
(an object of type {\tt Traffic})---in the following way:
{\small
\begin{verbatim}
void    Root::initTraffic () {

  SGroup *Snd, *Rcv;
  CGroup *Cgr;
  double MeanMIT, MeanMLE;

  readIn (MinPL);
  readIn (MaxPL);
  readIn (FrameL);
  readIn (MeanMIT);
  readIn (MeanMLE);

  Snd  = create SGroup (-1, &TheHub);
  Rcv  = Snd;
  Cgr  = create CGroup (Snd, Rcv);
  TPat = create TPattern (Cgr, MIT_exp+MLE_exp, MeanMIT, MeanMLE);

  readIn (EchoTimeout);
  readIn (HdrRecTime);
};
\end{verbatim} }

The function starts with reading traffic-related protocol parameters from
the input file.
Then, it creates a group of stations: this group consists of all the stations
in the network with exception of the {\tt Hub}, i.e.,
of all regular stations.
Note that the {\tt Hub} does not receive any messages to transmit from
outside: it only relays packets arriving from regular stations.

The only traffic pattern (pointed to by {\tt TPat}) created by the function
is based on a communication group consisting of two identical station groups,
each containing all the regular stations.
With this definition, all regular stations contribute the same amount of
traffic to the network load and the global traffic pattern is uniform.
Both the mean message inter-arrival time and the message length are
exponentially distributed with the mean values read from the input file.

The last operation performed by {\tt initTraffic} is reading the two delay values
{\tt EchoTimeout} and {\tt HdrRecTime}.

\subsection{The protocol code}

Each regular station runs three ``permanent" processes ({\tt Xmitter},
{\tt Receiver}, and {\tt Monitor}) and occasionally
spawns one additional process (called {\tt Trumpet})
which disappears after a while.

\subsubsection{The transmitter}

The {\tt Xmitter} process transmits packets on the station's selection port
and determines whether the transmission has been successful.
Its definition is listed below.
{\small
\begin{verbatim}
process Xmitter (User) {

  Port    *SPort;
  Packet  *Buffer;

  setup () {
      SPort  = S->SPort;
      Buffer = &(S->Buffer);
  };

  states {NewPacket, Retransmit, Done, Confirmed, Lost};

  perform {
    state NewPacket:
      if (Client->getPacket (Buffer, MinPL, MaxPL, FrameL))
        proceed Retransmit;
      Client->wait (ARRIVAL, NewPacket);
    state Retransmit:
      SPort->transmit (Buffer, Done);
      S->StartEW->put ();
      S->NACK->wait (RECEIVE, Lost);
    state Done:
      SPort->stop ();
      S->NACK->wait (RECEIVE, Retransmit);
      S->ACK->wait (RECEIVE , Confirmed);
    state Confirmed:
      Buffer->release ();
      proceed NewPacket;
    state Lost:
      SPort->abort ();
      proceed Retransmit;
   }
};
\end{verbatim} }

The two local attributes of the process, {\tt SPort} and {\tt Buffer}, are set
by the {\tt setup} function to identify the station's selection port
and the only packet buffer, respectively.

If a packet becomes ready for transmission, the transmitter wakes up at
state {\tt Retransmit}.
Having acquired a packet from the {\tt Client}, the process sends a signal to the
{\tt Monitor} (by putting a dummy item into the {\tt StartEW} mailbox)
and starts to transmit the packet on the station's selection port.
Note the reference to the {\tt S} attribute which
points to the station owning the process.
Thus, {\tt S->StartEW} identifies one of the mailboxes declared at the station.

Having notified the {\tt Monitor} about the beginning of a packet transmission,
the transmitter issues a wait request to the {\tt NACK}
mailbox for a possible ``negative acknowledgement" signal coming from the
{\tt Monitor}.
If the packet is long enough, it may happen that the
{\tt Monitor} decides to send the {\tt NACK} signal before the
packet has been entirely transmitted.
In such case, there is no point in continuing the transmission and the
process aborts the transfer.
Note that nothing will be lost if the ``positive acknowledgement"
signal arrives from the {\tt Monitor} before the transfer is complete.
The signal will simply remain pending in the {\tt ACK} mailbox
and it will be immediately {\tt RECEIVE}d when the
transmitter issues a wait request for it at state {\tt Done}.

\subsubsection{The monitor}

The purpose of the {\tt Monitor} is to detect echo timeouts and notify the
transmitter about the success or failure of the last transfer attempt.
The {\tt Monitor} is implemented with the help of an auxiliary process which
is created and killed by the {\tt Monitor} dynamically.
To understand the rationale of such a solution let us discuss the
expected behavior of the {\tt Monitor}.

The process has nothing to do until it is awakened by the
{\tt StartEW} signal from the transmitter.
Then, it has to set up an alarm clock for {\tt EchoTimeout} time units.
While waiting for the timer to go off, the {\tt Monitor} is supposed to
listen to the broadcast port.
Whenever a packet arrives there, the process must wake up, examine
the packet header to identify the sender, and, if the sender happens to
be different from the process' station, the waiting must continue until
either the right packet arrives or the alarm clock goes off.
One natural way to implement the alarm clock is to issue a
simple timer request.
Let us note, however, that whenever the {\tt Monitor} is interrupted by a
packet
on the broadcast port and finds out that the packet has been sent
by some other station, it
will have to set the alarm clock again for an appropriately updated time
interval.
Thus, the process will have to keep track of how much time it has been
waiting so far and how much time still remains to wait.

It seems to be a somewhat simpler approach to create a special process to
implement an independent alarm clock that can wait for the requested amount
of time without unsolicited interruptions.
This is the role of the {\tt Trumpet} process which is defined as follows:
{\small
\begin{verbatim}
process Trumpet {

  TIME  Delay;

  states {Start, Play};

  perform {
    state Start:
      Timer->wait (EchoTimeout, Play);
    state Play:
      S->Timeout->put ();
      terminate ();
   }
};
\end{verbatim} }

When the {\tt Trumpet} is created,
it issues a wait request to the timer for {\tt EchoTimeout}
time units.
When that delay elapses, the process is restarted at {\tt Play}
where it sends a timeout signal (by storing a dummy item in
the {\tt Timeout} mailbox) and terminates.

The {\tt Monitor} code looks as follows:
{\small
\begin{verbatim}
Monitor::perform {
    state WaitSignal:
      S->StartEW->wait (RECEIVE, WaitEcho);
    state WaitEcho:
      AClock = create Trumpet;
      proceed Waiting;
    state Waiting:
      BPort->wait (BOT, Packet);
      S->Timeout->wait (RECEIVE, NoEcho);
    state Packet:
      if (ThePacket -> Sender == ident (S)) {
        timer->wait (HdrRecTime, Echo);
        S->Timeout->wait (RECEIVE, NoEcho);
      } else
        skipto Waiting;
    state Echo:
      if (S->Timeout->erase () == 0) AClock -> terminate ();
      S->ACK->put ();
      proceed WaitSignal;
    state NoEcho:
      S->NACK->put ();
      proceed WaitSignal;
};
\end{verbatim} }

Upon reception of the {\tt StartEW} signal from the transmitter,
the {\tt Monitor} creates an instance of {\tt Trumpet} (pointed to by
{\tt AClock}).
Then the process examines all packets appearing on the station's broadcast port
until it either gets the {\tt Timeout} signal
or finds a packet sent by the current station.
Note that the {\tt Monitor} simulates the operation of recognizing the
packet's sender.
We assume that the sender can be determined {\tt HdrRecTime} time
units since the moment when the beginning of the packet was heard.
Intentionally, this delay corresponds to receiving (a part of) the
packet header.

If the awaited packet echo arrives before the alarm clock goes off,
the {\tt AClock} process is killed, but only if the {\tt Timeout} signal
is not already pending at the station.
The {\tt erase} method empties the mailbox and returns the number of
removed items.
The pending {\tt Timeout} signal means that {\tt AClock} has already killed
itself: the two events have occurred at the same {\em ITU}.

\subsubsection{The receiver}

The {\tt Receiver} process (run by a regular station)
is completely independent of the other processes:
its sole purpose is to listen to the broadcast port and detect packets
addressed to its owner.
{\small
\begin{verbatim}
Receiver::perform {
    state Wait:
      BPort->wait (EMP, Packet);
    state Packet:
      client->receive (ThePacket, BPort);
      skipto Wait;
};
\end{verbatim} }

When started for the first time, the process
issues a {\tt wait} request to the broadcast port specifying that it wants
to be restarted by the earliest {\tt EMP} event---the end of a packet
addressed to its station.
This corresponds to the complete reception of a packet and is the only
interesting event from the viewpoint of the {\tt Receiver}.
Upon reception of a packet, the process calls the {\tt Client}'s method
{\tt receive} that updates some performance measures.

Having received the packet, the process is ready to await the arrival of the
next one.
However, before branching back to {\tt Wait}, the {\tt Receiver}
has to make sure
that the last {\tt EMP} event has disappeared from the port.
This is the reason for using {\tt skipto} instead of {\tt proceed}.
Otherwise, the process would loop infinitely on the same event:
the modeled time does not flow automatically
while a process is running.

\subsubsection{The {\em Hub\/} process}

The {\em Hub\/} station runs {\tt NUsers} identical copies of the same process,
each copy servicing one selection port.
The type of this process is declared as:
{\small
\begin{verbatim}
process HubProcess (Hub) {

  Port  *SPort, *BPort;

  void setup (int pn) {
      SPort = S->HPorts [pn];
      BPort = S->HPorts [0];
  };

  states {Wait, NewPacket, Done};

  perform {
    state Wait:
      SPort->wait (BOT, NewPacket);
    state NewPacket:
      if (S->Status == Busy) skipto Wait;
      S->Status = Busy;
      BPort->transmit (ThePacket, Done);
    state Done:
      BPort->stop ();
      S->Status = Idle;
      proceed Wait;
  };
};
\end{verbatim} }

Each of the {\em Hub\/} processes has access to one selection port
and to the broadcast port.
The selection port serviced by the process is specified
as the {\tt setup} parameter when the process is created (see below).

The {\tt Hub} process awaits a packet arrival on its
selection port (event {\tt BOT}).
If the packet arrives while the broadcast medium is busy (the {\em Hub} is
already relaying a packet from another selection port), the packet
is ignored (the {\tt BOT} event is {\em skipped\/}) and the process
awaits another packet arrival.
If the broadcast channel is idle, it is immediately marked as busy and the
packet is re-transmitted on the {\tt Hub}'s broadcast port.
When the transfer is complete, the {\tt Hub} status is reset to idle and the
process resumes waiting for another packet to relay.

\subsubsection{Protocol startup}

The last element of the startup action performed by the
{\tt Root} process is the creation of the protocol processes;
this part is done by the following function:
{\small
\begin{verbatim}
void    Root::initProtocol () {

  for (int i = 0; i < NUsers; i++) {
    TheStation = idToStation (i);
    create Xmitter;
    create Receiver;
    create Monitor;
  }
  TheStation = TheHub;
  for (i = 1; i <= NUsers; i++) create HubProcess (i);
};
\end{verbatim} }

With the first loop, the three processes are created for each regular station.
Before the {\tt create} operations are issued, the context variable
{\tt TheStation} is set to point to the station object which is supposed to
own the created processes.
The {\tt Hub} runs {\tt NUsers} versions of the same {\tt HubProcess}.
The parameter passed to the {\tt setup} function of this process identifies
one of the {\tt Hub}'s selection ports.

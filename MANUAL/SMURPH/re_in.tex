\section{Introduction}

The design of a low-level communication protocol must deal with very
specific physical phenomena occurring in the communication channels and
the hardware used to interconnect them.
Problems like race conditions, signal propagation time, the limited
accuracy of independent clocks, etc., are typical examples of issues
that must be accounted for by the protocol designer.
These issues become even more critical with the advancing technology:
high-speed networks call for protocols that can explore to the maximum
possible extent the advantageous physical properties of the underlying hardware
components.

On the other hand, there seems to be a substantial gap in the assortment of
the specification tools available for aiding the protocol design process.
Most researchers in this area focus their attention on higher protocol
layers which are removed from the hardware far enough to allow for a
reasonable degree of abstraction.
In particular, the known protocol specification and verification
tools as {\em ESTELLE\/}
(\cite{bud87,est87}), {\em LOTOS\/} (\cite{bob87,lob88}), or
{\em PROMELA/SPIN\/} (\cite{hol91}), although
useful for describing high-level logical aspects of communication,
fail to capture some very important low-level aspects of distributed systems,
e.g., the accurate timing of events.

As far as performance evaluation is concerned, the situation is not much better.
Approximate analytical methods have exhibited their limitations sufficiently
many times to be treated with a limited confidence (cf.\ 
\cite{bmk88,mis86}).
Accurate analytical models exist for very few simple networks and protocols
and there is no straightforward way to generalize them into more complex
systems.
General-purpose simulation packages are not of much help, either.
An accurate description of the behavior of a multi-point broadcast channel
requires about the same effort in a regular programming language
as in a language oriented for simulation.
The most tricky part of this description is not modeling events, but 
predicting their timing.
The user of a protocol simulator is not excited about being able to
schedule events.
The problem is: {\em how to schedule these events\/} and, in our opinion,
a good protocol modeling system should absorb this problem completely.
The interface offered to the user should resemble (in terms of its
functionality) a realistic physical environment for protocol execution.

\smurph\ is an object-oriented software package for modeling discrete events
in communication protocols.
It can be viewed as both an implementation of a certain protocol specification
language and a process-driven simulator oriented towards investigating
medium access control (MAC) level protocols.
In accordance with the above postulate,
the structure of the simulator is completely invisible to the user.
All the simulation-related operations, as creating and scheduling
individual events, maintaining a consistent notion of time, etc.,
are covered by a high-level interface.
A protocol description in \smurph\ looks like a program that
could be executed on a hypothetical hardware.
\smurph\ emulates this hardware and thus provides a realistic environment
for executing protocols described in its specification language.

Protocol execution in \smurph\ can be monitored.
One reason for monitoring the behavior of a protocol is to investigate its
performance by gathering empirical data.
Although \smurph\ does not purport to be a protocol verification system,
it offers some tools for protocol testing.
These tools, the so-called {\em observers}, look like programmable dynamic
assertions describing legitimate sequences of protocol actions.

\smurph\ has been programmed in C++.
The package is configurable: it is not a single interpreter for a variety of
protocols, but it configures itself into a stand-alone modeling program
for each particular application.

Although the protocol description language of \smurph\ is essentially C++,
the standard data types, objects, functions, and macros provided by the
package extend this language substantially.
With this approach, the user gets the full power of C++ combined
with the power of a realistic, emulated environment for programming and
executing communication protocols.

\smurph\ descends from its predecessor, called \lansf (\cite{gbr90a}),
which was designed
and implemented by the authors in 1987.
\lansf\ was programmed in C and has evolved a number of times from its
original version.
It has been successfully applied to investigating the performance and
correctness of a number of protocols for local area networks
(\cite{dog88,dog89,dgr90,dgr90a,gbr87,gbr87a,gbr89a,gbr88c}) and
other distributed systems.

One disadvantage of \lansf\ was its low flexibility in describing
compound data structures.
The idea of re-implementing \lansf\ in C++ was born from our collaboration
with the networking group at the Lockheed Space and Missile Company.
At first, Objective C was tried, but after a brief preliminary design we
decided to use C++ instead.
C++ seems to capture all the essential features of Objective C, and at the same
time is devoid of some deficiencies of that language.
